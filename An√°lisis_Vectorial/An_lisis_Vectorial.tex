% Definición tipo de documento
\documentclass[12pt,openany]{book}

% Paquetes
\usepackage{lmodern}
\usepackage[T1]{fontenc}
\usepackage[spanish,activeacute]{babel}
\usepackage{mathtools}
\usepackage{amsmath}
\usepackage{amssymb}
\usepackage[latin1]{inputenc}

%Para hacer cajas bonitas
\usepackage{fancybox}

%Permite crear columnas en el documento
\usepackage{multicol} 
\usepackage{color}
\usepackage{comment}

%Permite incluir imagenes
\usepackage{graphicx}
\graphicspath{{images/}} %Indicamos la ruta de las imagenes

% Configuracion de encabezado y pie de pagina
\usepackage{fancyhdr}                                          
    \pagestyle{fancy}                                               
    \setlength{\headheight}{16pt}                                   
    \setlength{\parskip}{0.5em}                                    
    \renewcommand{\footrulewidth}{0.5pt}                            
    \lhead{ }

\title{An\'alisis Vectorial}
\author{Nico\'as Sayago Abigail}

% ////////////////////////////////////////
% ///////// INICIO DEL DOCUMENTO /////////
%/////////////////////////////////////////

\begin{document}
    \begin{titlepage}
        \centering
        \rule{\linewidth}{0.5mm} \\[1.0cm]
            { \huge \bfseries An\'alisis Vectorial}\\[1.0cm] 
        \rule{\linewidth}{0.5mm} \\[2.0cm]
        \centering
        Abigail Nicol\'as Sayago 
    \end{titlepage}
   

\tableofcontents

% /////////////////////////
% //// INTRODUCCION //////
% ///////////////////////
\chapter*{Introducci\'on}
\addcontentsline{toc}{chapter}{Introducci\'on} \markboth{INTRODUCTION}{}
	Este sencillo cuadernillo de apuntes ha sido creado con la finalidad de que disfrutes m\'as 
	de tu maravillosa clase de Análisis Vectorial. Los apuntes fueron tomados en la clase del 
	profesor Miguel Olvera Aldana en la ESCOM, me he esforzado por evitar que tenga errores...


% /////////////////////////////////////
% /// CAPITULO Análisis Vectorial ////
% ///////////////////////////////////
\chapter{\'Algebra Vectorial}

	% /////////////////////////////////////
	% /// Sección Escalares y vectores////
	% ///////////////////////////////////
	\section{Escalares y vectores}
	    
	    % /////////////////////////////////////////////////
	    % //// Cantidadades escalares y vectoriales //////
	    % ///////////////////////////////////////////////
	    \subsection{Cantidades escalares y vectoriales}
	    Nosotros ya sabemos que existen ciertas cantidades que se pueden describir 
	    totalmente con un solo n\'umero, como por ejemplo la masa, densidad e incluso 
	    el tiempo, sin embargo tambi\'en existen otras que deben tener una 
	    \textsl{direcci\'on} como el desplazamiento o la velocidad por
	     mencionar solo algunas.
	    
	    Ahora bien, a las primeras cantidades se les llama \textbf{escalares} y a la
	    segunda \textbf{vectoriales}, como nuestro curso es de An\'alisis vectorial 
	    por obvias razones utilizaremos cantidades vectoriales, entonces centr\'emonos en ellas. 
	    Est\'as cantidades incluyen una \textsl{magnitud} es decir "qu\'e tan grande es".
	    		
		% /////////////////////////////////
	    % //// Definicion de vector //////
	    % ///////////////////////////////
	    \subsection{Definici\'on de vector}
	    En pocas y reducidas palabras un vector es un ente matem\'atico que \textsl{vive} 
	    en un espacio vectorial. 

		Los vectores se expresan con una flecha y eso no es porque a alguien se le 
	    ocurri\'o, sino m\'as bien porque es necesario diferenciarlos de las 
	    cantidades escalares puesto que tienen propiedades diferentes. Entonces
	    en este documento veremos algo as\'i: $\vec{A}$.
	    
	    \noindent \textbf{JAM\'AS lo olvides, los vectores llevan flecha.}  

		    % ////////////////////////////////////
		    % /////// Tipos de Vectores /////////
		    % //////////////////////////////////
		    \subsubsection{Tipos de Vectores}
		    As\'i es, existen diferentes tipos de vectores, ahora veamos algunos 
		    vectores que muy probablemente t\'u ya hab\'ias visto pero no sab\'ias
		    que era un vector.
		    
		    \noindent Matrices:
		        $$\vec{A}=
		        \left(
		            \begin{smallmatrix}
		            a & h & g\\
		            h & b & f\\
		            g & f & c
		            \end{smallmatrix}
		        \right)
		        $$
		   \noindent Polinomios:\\
		        Con $a \epsilon \mathbb{R}$
		        $$\vec{P}=
		        a_n X^n + a_{n-1} X^{n-1} + ... + a_1 X + a 
		        $$
		    \noindent Funciones:
		        $$\vec{F(x)}= A\sin{x} + B\cos{x}$$
		    \noindent Vector n-ada:\\
		        Con $X \epsilon \mathbb{R}$
		        $$\vec{X}=
		        (X_1, X_2, X_3,..., X_n)
		        $$
		
	    % ////////////////////////////////////////////////////////////
	    % //// Componentes cartesianas y magnitud de un Vector //////
	    % //////////////////////////////////////////////////////////
	    \subsection{Representaci\'on en componentes cartesianas y magnitud de un vector}    
	    
	    El \'ultimo tipo de vector que vimos, llamado n-ada solo se puede dibujar hasta 
	    $\mathbb{R}^3$. 

	    % ////////////////////////////////////
	    % /////// Tipos de Vectores /////////
	    % //////////////////////////////////
	    \subsection{Vectores unitarios}

   	    % /////////////////////////////////////////////////////
	    % /////// Dependencia e independencia lineal /////////
	    % ///////////////////////////////////////////////////
	    \subsection{Dependencia e independencia lineal}

	    \textsl{Definici\'on.} Sean $ \left\lbrace \vec{X_{1}}, \vec{X_{2}}, ... ,
	    							\vec{X_{n}} \right\rbrace $ un conjunto de vectores, entonces:
	    							$$ \alpha_{1} \vec{X_{1}}+\alpha_{2} \vec{X_{2}}+
	    							...+\alpha_{n} \vec{X_{n}} = \vec{0} $$
	    							es un conjunto de vectores linealmente independientes (\textsl{l.i}) 
	    							si: $$\alpha_{1} = \alpha_{2} = ... = \alpha_{n} = 0 $$\\
		\textbf{Ejemplo.} Demostrar que $\lbrace\hat{i}, \hat{j}, \hat{k}\rbrace $ es \textsl{l.i}\\
	    \noindent Sabemos que:
	    \begin{multicols}{3}
	    $  \hat{i} = (1,0,0) $
        \columnbreak
        
	    $  \hat{j} = (0,1,0) $
	   \columnbreak
	   
	   $   \hat{j} = (0,0,1) $
	    \end{multicols}

	    \noindent Sustituyendo en la definici\'on tenemos que:
	    \begin{equation*}
	    	\begin{split}
				\alpha \hat{i} + \beta \hat{j} + \gamma \hat{k}	
				&= (\alpha, 0,0) + (0,\beta,0) + (0,0,\gamma)\\
				&= (\alpha, \beta, \gamma)		\\	
				&=(0,0,0)
			\end{split}
		\end{equation*}

	   \noindent Recuerda... \textbf{"Dos vectores son iguales, si y solo si componente a componente son iguales"} 
       por lo tanto tenemos lo siguiente:

       \begin{multicols}{3}
       $ \alpha = 0 $
       \columnbreak
       
       $ \beta = 0  $
       \columnbreak
       
       $ \gamma = 0 $
       \end{multicols}

       $ \therefore \lbrace\hat{i}, \hat{j}, \hat{k} \rbrace $ son \textsl{l.i}

	% //////////////////////////
	% /// Algebra Vectorial////
	% ////////////////////////
	\section{\'Algebra Vectorial}   

	    % //////////////////////////////////////
	    % /////// Axiomas vectoriales /////////
	    % ////////////////////////////////////
	    \subsection{Axiomas vectoriales}

		    \begin{itemize}
	      
		        \item Cerradura bajo la suma
		      
		         Si $\vec{x}$ $\epsilon$ $\vec{V}$ y 
		         $\vec{y}$ $\epsilon$ $\vec{V}$,entonces 
		         $\vec{x} + \vec{y}$ $\epsilon$ $\vec{V}$
		         
		        \item Ley asociativa de la suma de Vectores
		        
		         Para todo $\vec{x}$, $\vec{y}$ y $\vec{z}$ 
		         en $\vec{V}$, entonces ($\vec{x}$ $+$ $\vec{y}$) 
		         $+$ $\vec{z}$ $=$ $\vec{x}$ + ($\vec{y}$ + 
		         $\vec{z}$)
		         
		        \item Ley conmutativa de la suma de Vectores
		        
		         Si $\vec{x}$ y $\vec{y}$ en 
		         $\vec{V}$, entonces $\vec{x}$ + $\vec{y}$ 
		         $=$ $\vec{y}$ + $\vec{x}$ 
		         
		        \item Vector cero o Id\'entico aditivo
		        
		        Si existe un $\vec{0}$ $\epsilon$ $\vec{V}$ tal que
		        para todo $\vec{x}$ $\epsilon$ $\vec{V}$,
		        $\vec{x}$ + $\vec{0}$ $=$  $\vec{x}$
		        
		        \item  Inverso aditivo
		        
		        Si existe un $\vec{x}$ $\epsilon$ $\vec{V}$ tal 
		        que para todo $\vec{-x}$ $\epsilon$ $\vec{V}$,
		         $\vec{x}$ + ($\vec{-x}$) $=$  $\vec{0}$
		        
		        \item Cerradura bajo la multiplicaci\'on por un escalar
		        
		        Si existe un $\vec{x}$ $\epsilon$ $\vec{V}$ y 
		        $\alpha$ es un escalar, entonces $\alpha\vec{V}$ $\epsilon$
		         $\vec{V}$
		        
		        \item Primera Ley distributiva
		        
		        Si $\vec{x}$ $\epsilon$ $\vec{V}$ y $\alpha$ es 
		        un escalar, entonces $\alpha$ ($\vec{x}$ + $\vec{y}$)
		        = $\alpha$ $\vec{x}$ + $\alpha$ $\vec{y}$
		        
		        \item Segunda Ley distributiva
		        
		         Si $\vec{x}$ $\epsilon$ $\vec{V}$ y $\alpha$ y $\beta$
		        son escalares, entonces ($\alpha$ + $\beta$) $\vec{x}$ =  $\alpha$
		        $\vec{x}$ + $\beta$ $\vec{x}$
		         
		        \item Ley asociativa de la multiplicaci\'on por escalares
		        
		        Si $\vec{x}$ $\epsilon$ $\vec{V}$ y $\alpha$ y $\beta$ 
		        son escalares, entonces $\alpha$ ($\beta$ $\vec{x}$) = $\beta$ 
		        ($\alpha$ $\vec{x}$)
		        
		        \item Para cada vector $\vec{x}$ $\epsilon$ $\vec{V}$
		        existe ($1$) $\vec{x}$ = $\vec{x}$
	    	
	        \end{itemize}

	   	% //////////////////////////////////////////////////////////////////
	    % /////// Adicion y sustracion de vectores y aplicaciones /////////
	    % ////////////////////////////////////////////////////////////////
	    \subsection{Adici\'on y substracci\'on de vectores y aplicaciones}

	    	\subsubsection{Definici\'on Suma de vectores}
		    	 \noindent Sean $ \vec{X} = ( x_{1}, x_{2}, x_{3}, ..., x_{n})$ y $\vec{Y} = ( y_{1}, y_{2}, y_{3}, ..., y_{n}) $ Con  $x_{i}$ \& $y_{i} \epsilon \mathbb{R}$ \\
		    	entonces 
		    	$$\vec{X} + \vec{Y} = (x_{1} + y_{1} , x_{2} + y_{2}, ... , x_{n} + y_{n} )$$
	    	\subsubsection{Ejemplo}
	    	Sumar $\vec{A} + \vec{B}$ , $\vec{A} + \vec{C}$ ,
	    	$\vec{A} + \vec{B} + \vec{C}$, si:

	    	\begin{multicols}{3}
       			$\vec{A}=(1,1)$
       			\columnbreak

       			$\vec{B}=(3,-5)$
       			\columnbreak
       			
       			$\vec{C}=(-1,-2)$
	    	\end{multicols} 
	    	\noindent\textsl{Soluci\'on}:\\
	    	\begin{itemize}
		    	%AQUI DEBE IR LA IMAGEN DEL VECTOR RESULTANTE DEL LADO IZQUIERDO
		    	\item Ejercicio 1
		    	\noindent Por definici\'on tenemos que:
		    	\begin{equation*}
			    	\begin{split}
						\vec{A}+\vec{B} &=(1,1) + (3,-5) 
		    			=(1 + 3 , 3 + (-5)) 
		    			=(4, -4)
					\end{split}
				\end{equation*}

				\noindent Obteniendo el m\'odulo:\\
				\begin{equation*}
			    	\begin{split}
						|\vec{A}+\vec{B}|= \sqrt{(4)^{2}+(-4)^{2}}
						=\sqrt{2\cdot16}
						=4\sqrt{2}
					\end{split}
				\end{equation*}

				\noindent Obteniendo el argumento:\\
				\begin{equation*}
			    	\begin{split}
			    		\theta = \frac{3}{2}\pi + \alpha
			    		=\frac{3}{2}\pi + \frac{1}{4}\pi
			    		=\frac{7}{4}\pi
					\end{split}
				\end{equation*}

				\item Ejercicio 2
		    	\noindent Por definici\'on tenemos que:
		    	\begin{equation*}
			    	\begin{split}
						\vec{A}+\vec{C} &=(1,1) + (-1,-2) 
		    			=(1 +(-1) , 1 + (-2)) 
		    			=(0, -1)
					\end{split}
				\end{equation*}

				\noindent Obteniendo el m\'odulo:\\
				\begin{equation*}
			    	\begin{split}
						|\vec{A}+\vec{C}|= \sqrt{(0)^{2}+(-1)^{2}}
						=1
					\end{split}
				\end{equation*}

				\noindent Obteniendo el argumento:\\
				\begin{equation*}
			    	\begin{split}
			    		\theta = \frac{3}{2}\pi
					\end{split}
				\end{equation*}

				\item Ejercicio 3
		    	\noindent Por definici\'on tenemos que:
		    	\begin{equation*}
			    	\begin{split}
						\vec{A}+\vec{B}+\vec{C} &=(1,1)+(3,-5)+ (-1,-2) \\
		    			&=(1 +3+(-1) , 1 +(-5)+(-2)) \\
		    			&=(3, -6)
					\end{split}
				\end{equation*}

				\noindent Obteniendo el m\'odulo:\\
				\begin{equation*}
			    	\begin{split}
						|\vec{A}+\vec{B}+\vec{C}|= \sqrt{(3)^{2}+(-6)^{2}}
						=\sqrt{9+36}
						=\sqrt{45}
						=3\sqrt{5}
					\end{split}
				\end{equation*}

				\noindent Obteniendo el argumento:\\
				\begin{equation*}
			    	\begin{split}
			    		\theta = \frac{3}{2}\pi + \alpha
			    		=\frac{3}{2}\pi + \arctan\frac{1}{2}
					\end{split}
				\end{equation*}
			\end{itemize}
			
			\subsubsection{Ejemplo}
			Un solido de 100 N de pero tiende del centro de una cuerda como se observa en la figura. 
			Hallar T de la figura.\\
			\noindent\textsl{Soluci\'on}:\\
			Recuerda observar el dibujo, de ah\'i podemos concluir:
			$$\vec{T_{1}}+\vec{T_{2}}+\vec{W}=\vec{0}$$
			\noindent De igual forma por la figura podemos obtener las componentes de $\vec{T_{1}}$ y 
			$\vec{T_{2}}$:
			\begin{equation*}
				\begin{split}
					\vec{T_{1}}=(-T_{1}\cos30^{\circ}, T_{1}\sin30^{\circ})\\
					\vec{T_{2}}=(T_{2}\cos30^{\circ}, T_{2}\sin30^{\circ})
				\end{split}
			\end{equation*}
			Ahora por definici\'on de la suma de vectores:
			\begin{equation*}
				\begin{split}
					\Rightarrow(-T_{1}\cos30^{\circ}+T_{2}\cos30^{\circ}+0 ,  
					T_{1}\sin30^{\circ}+T_{2}\sin30^{\circ}+100)
				\end{split}
			\end{equation*}
			\noindent Por ser la misma cuerda podemos decir:
				$$(0,2T\sin30^{\circ}-100)=(0,0)$$	
			\noindent Y como, dos vectores son iguales si y solo si componente a componente son iguales, entonces:
				$$2T\sin30^{\cirl}-100=0$$
			\noindent Obtenemos T:
				$$
					T=\frac{100}{2\sin30^{\circ}}=100
				$$

	   	% ///////////////////////////////////
	    % /////// Producto escalar /////////
	    % /////////////////////////////////
	    \subsection{Producto escalar}
	    Antes de dar la definici\'on oficial, debo decir que el producto punto, tambi\'en es conocido
	    como producto punto. Ahora s\'i...\\
	    \noindent\textsl{Definici\'on 1}.\\
	    \noindent Sea $\vec{A}$ y $\vec{B}$ dos vectores distintos del
	    vector cero, entonces 
	    $$ \vec{A}\cdot\vec{B}=|\vec{A}| |\vec{B}|\cos\theta $$
	    donde $\theta$ es en \'angulo formado por $\vec{A}$ y $\vec{B}$.	   	

	    \noindent\textsl{Definici\'on 2}.\\
	    
	    Sea 
	    $ \vec{A}=(A_{1}, A_{2}, A_{3}) $ \& $ \vec{B}=(B_{1}, B_{2}, B_{3}) $

	    \noindent Entonces:
	    \begin{equation*}
	    	\begin{split}
	    		\vec{A}\cdot\vec{B}=A_{1}B_{1}+A_{2}B_{2}+A_{3}B_{3}
	    	\end{split}
	    \end{equation*} 
	    \subsubsection{Delta de Kronecker}
		    Ahora observemos lo siguiente, anteriormente vimos los vectores unitarios, entonces, sea:
		    \begin{equation*}
		    	\begin{split}
		    		\vec{A}=A_{1}\hat{i}+A_{2}\hat{j}+A_{3}\hat{k}\\
					\vec{B}=B_{1}\hat{i}+B_{2}\hat{j}+B_{3}\hat{k}
		    	\end{split}
		    \end{equation*}
		    Presta atenci\'on a lo siguiente:
		    \begin{equation*}
		    	\begin{split}
		    		\vec{A}\cdot\vec{B}\Rightarrow
		    		\hat{i}\cdot\hat{i}=\hat{j}\cdot\hat{j}=\hat{z}\cdot\hat{z}=1
		    	\end{split}
		    \end{equation*}	    
		    Y haciendo:
		    \begin{equation*}
		    	\begin{split}
		    		\vec{A}\cdot\vec{B}\Rightarrow
		    		\hat{i}\cdot\hat{j}=\hat{i}\cdot\hat{k}=\hat{j}\cdot\hat{k}=0
		    	\end{split}
		    \end{equation*}

		    \noindent A lo anterior es conocido como la \textsl{Delta de Kronecker}, b\'asicamente es:
		    \begin{equation*}
		    	\delta_{ij}=
		    	\left\{
		    	\begin{aligned}
		    		1  \text{ si }& i=j\\
		    		0   \text{ si }& i\neq j
		    	\end{aligned}
		    	\right
		    \end{equation*}
		\subsubsection{Propiedad}
			Aplicando la definici\'on del producto escalar:\\
			\begin{equation*}
				\vec{A}\cdot\vec{A}=A_{1}^{2} + A_{2}^{2} + A_{3}^{2} =A^{2}	
			\end{equation*}
			Por otro lado:\\
			$$|\vec{A}|=\sqrt{A_{1}^{2} + A_{2}^{2} + A_{3}^{2} =A^{2}}$$
			Elevando al cuadrado la expresi\'on anterior:\\
			$$|\vec{A}|^{2}=A_{1}^{2} + A_{2}^{2} + A_{3}^{2}$$
			Por lo tanto, concluimos:
			$$|\vec{A}|^{2}=A^{2}$$

		% //////////////////////////////////////////////////////////
	    % /////// Problemas clasicos del producto escalar /////////
	    % ////////////////////////////////////////////////////////
		\subsection{Problemas Cl\'asicos del producto escalar}
		
			\subsubsection{Deducci\'on de la Ley de los cosenos}
				\noindent Deducir la Ley de los cosenos\\
				\noindent\textsl{Soluci\'on}:\\
				Observando la imagen tenemos:
				$$\vec{b}+\vec{c}=\vec{a}$$
				Para este problema utilizaremos el $\vec{c}$ por lo tanto:\\
				$$\vec{c}=\vec{a}-\vec{b}$$
				Multiplicamos por $\vec{c}$:\\
				$$
					\vec{c}\cdot\vec{c}=c^{2}
				$$
				Sustituyendo el valor de $\vec{c}$
				Tenemos que:\\
				\begin{equation*}
					\begin{split}
						c_{2}&=(\vec{a}-\vec{b})\cdot(\vec{a}-\vec{b})\\
							&=\vec{a}\cdot\vec{a}-\vec{a}\cdot\vec{b}-
							\vec{b}\cdot\vec{a}+\vec{b}\cdot\vec{b}				
					\end{split}
				\end{equation*}
				Sabemos que el producto interno conmuta, entonces:\\
				$$c^{2}=a^{2}-2\vec{a}\cdot\vec{b}+b^{2} $$
				Por la definici\'on 1:\\
				$$c^{2}=a^{2}+b^{2}-2ab\cos\gamma $$
			\subsubsection{Proyecci\'on de vectores}
				\noindent Encuentre la proyecci\'on $\vec{A}=(\hat{i}-2\hat{j}+3\hat{k})$ sobre el vector 
				$(\vec{B}=\hat{i}+2\hat{j}+2\hat{k})$
				\noindent\textsl{Soluci\'on}:\\
				Recordando la Definici\'on:
				$$ \vec{X}=|\vec{X}|\hat{x}$$
				Obtenemos $\hat{b}$:
				\begin{equation*}
					\begin{split}
						\hat{b}&=\frac{(1,2,2)}{\sqrt{1+4+4}}=\frac{(1,2,2)}{(3)}
					\end{split}
				\end{equation*}
				Ahora que tenemos $\hat{b}$, para la proyecci\'on de $\vec{A}$ sobre $\vec{B}$
				\begin{equation*}
					\begin{split}
						Proy_{\vec{B}}\vec{A}=\vec{A}\cdot\hat{b}
																   =(1,-2,3)\cdot\frac{1,2,2}{3}
																   =\frac{(1-4+6)}{3}
																   =1
					\end{split}
				\end{equation*}
			\subsubsection{Demostracion 1}
				\noindent Demuestra que $\cos(\theta-\beta)=\cos\theta\cos\beta+\sin\theta\sin\beta$\\
				\noindent\textsl{Soluci\'on}:\\
				Observando el dibujo sabemos que:
				\begin{equation}
					\begin{split}
						\vec{A}=(A\cos\theta,A\sin\theta)\\
						\vec{B}=(B\cos\beta,B\sin\beta)\\
					\end{split}
				\end{equation}
				De la definici\'on 1 del producto escalar, tenemos:\\
				\begin{equation}
					\begin{split}
						\vec{A}\cdot\vec{B}=AB\cos(\theta-\beta)
						\label{def_1}
					\end{split}
				\end{equation}
				De la definici\'on 2 del producto escalar, tenemos:\\
				\begin{equation*}
					\begin{split}
						\vec{A}=(A\cos\theta\hat{i} + A\sin\theta\hat{j})\\
						\vec{B}=(B\cos\beta\hat{i} + B\sin\beta\hat{j})\\
					\end{split}
				\end{equation*}
				Haciendo el producto escalar de $\vec{A}$ y $\vec{B}$, tenemos:
				\begin{equation}
					\begin{split}
						\vec{A}\cdot\vec{B}&=AB\cos\theta\cos\beta + AB\sin\theta\sin\beta\\
						 										 &=AB(\cos\theta\cos\beta + \sin\theta\sin\beta)
						\label{produc}
					\end{split}
				\end{equation}
				Con las ecuaciones \ref{def_1} $=$ \ref{produc}:
				\begin{equation*}
					\begin{split}
						AB\cos(\theta-\beta) &=AB(\cos\theta\cos\beta + \sin\theta\sin\beta)\\
						\Rightarrow \cos(\theta-\beta) &=\cos\theta\cos\beta + \sin\theta\sin\beta
					\end{split}
				\end{equation*}

			\subsubsection{Demostracion 2}
				\noindent Demuestra que $\cos(\alpha+\beta)=\cos\alpha\cos\beta-\sin\alpha\sin\beta$\\
				\noindent\textsl{Soluci\'on}:\\
				Observando el dibujo sabemos que:
				\begin{equation*}
					\begin{split}
						\vec{A}=(A\cos\alpha,A\sin\alpha)\\
						\vec{B}=(B\cos\beta,-B\sin\beta)\\
					\end{split}
				\end{equation*}
				De la definici\'on 1 del producto escalar, tenemos:\\
				\begin{equation}
					\begin{split}
						\vec{A}\cdot\vec{B}=AB\cos(\alpha+\beta)
						\label{def_11}
					\end{split}
				\end{equation}
				De la definici\'on 2 del producto escalar, tenemos:\\
				\begin{equation*}
					\begin{split}
						\vec{A}=(A\cos\alpha\hat{i} + A\sin\alpha\hat{j})\\
						\vec{B}=(B\cos\beta\hat{i} - B\sin\beta\hat{j})\\
					\end{split}
				\end{equation*}
				Haciendo el producto escalar de $\vec{A}$ y $\vec{B}$, tenemos:
				\begin{equation}
					\begin{split}
						\vec{A}\cdot\vec{B}&=AB\cos\alpha\cos\beta - AB\sin\alpha\sin\beta\\
						 										 &=AB(\cos\alpha\cos\beta - \sin\alpha\sin\beta)
						\label{produc2}
					\end{split}
				\end{equation}
				Con las ecuaciones \ref{def_11} $=$ \ref{produc2}:
				\begin{equation*}
					\begin{split}
						AB\cos(\alpha+\beta)&=AB(\cos\alpha\cos\beta - \sin\alpha\sin\beta)\\
						\Rightarrow \cos(\alpha+\beta)&=\cos\alpha\cos\beta - \sin\alpha\sin\beta
					\end{split}
				\end{equation*}			

			\noindent\textbf{IMPORTANTE:}\\
				Considero importante la siguiente conclusi\'on porque, por lo menos en mi opini\'on, es un
				hermoso ejemplo de lo maravillosas que son las matem\'aticas. 
				De las dos demostraciones anteriores podemos concluir lo siguiente:\\
				\begin{center}
					\doublebox{$\cos(\alpha\pm\beta)=\cos\alpha\cos\beta \pm \sin\alpha\sin\beta$}\\
				\end{center}
				Ahora supongamos que $\alpha=\beta=\theta$
				\begin{multicols}{2}
				\columnseprulecolor{\color{black}}
		    	\setlength{\columnseprule}{1pt}
					\begin{center}
						\begin{equation*}
							\begin{split}
								\cos(2\theta)&=\cos^{2}\theta-\sin^{2}\theta\\
								1=\cos(0)&=\cos^{2}\theta-\sin^{2}\theta\\
								1+\cos(2\theta)&=2\cos^{2}\theta
							\end{split}
						\end{equation*}
						Por lo tanto:
						$$
							\cos^{2}\theta=\frac{1+\cos(2\theta)}{2}	
						$$
				\breakcolumn
						\begin{equation*}
								\begin{split}
									\cos(2\theta)&=\cos^{2}\theta-\sin^{2}\theta\\
									-1&=-\cos^{2}\theta-\sin^{2}\theta\\
									\cos(2\theta)-1&=-2\sin^{2}\theta
								\end{split}
						\end{equation*}
						Por lo tanto:
						$$
							\sin^{2}\theta=\frac{1-\cos^{2}\theta}{2}	
						$$
					\end{center}
				\end{multicols}
				
			\subsubsection{\'Angulo agudo en un cubo}
				Calcule el \'angulo agudo que formas dos diagonales de un cubo.\\
				\noindent\textsl{Soluci\'on}:\\
				De la figura tenemos:
				\begin{equation*}
					\begin{split}
						\vec{A}&=(3,3,3)\\
						\vec{B}&=(-3,-3,3)\\
					\end{split}
				\end{equation*}
				Obtenemos el m\'odulo de $\vec{A}$ y $\vec{B}$
				\begin{equation*}
					\begin{split}
						|\vec{A}|&=\sqrt{(3^{2})+(3^{2})+(3^{2})}\\
											&=\sqrt{9+9+9}=\sqrt{27}=3\sqrt{3}					
					\end{split}
				\end{equation*}
				\begin{equation*}
					\begin{split}
						|\vec{B}|&=\sqrt{(-3^{2})+(-3^{2})+(3^{2})}\\
											&=\sqrt{9+9+9}=\sqrt{27}=3\sqrt{3}					
					\end{split}
				\end{equation*}
				Por la definici\'on 1 del producto escalar:
				\begin{equation}
					\begin{split}
						\vec{A}\cdot\vec{B}=27\cos\alpha					
					\end{split}
					\label{Def1}
				\end{equation}
				Por la definici\'on 2 del producto escalar:
				\begin{equation}
					\begin{split}
						\vec{A}\cdot\vec{B}=-9-9+9=-9					
					\end{split}
					\label{Def2}
				\end{equation}
				Igualando los resultados de las ecuaciones \ref{Def1} y \ref{Def2} tenemos:
				\begin{equation*}
					\begin{split}
						27\cos\alpha&=-9\\
						\cos\alpha&=-\frac{9}{27}=-\frac{1}{3}\\
						&\Rightarrow \alpha=\arccos(-\frac{1}{3})
					\end{split}
				\end{equation*}
			\subsubsection{Trabajo}
				C\'alcule el trabajo realizado para mover un objeto a lo largo de un
				vector $\vec{r}=3\hat{i}+\hat{j}-5\hat{k}$, si se le aplica
				la $\vec{F}=2\hat{i}-\hat{j}-\hat{k}$.\\
				\noindent\textsl{Soluci\'on}:\\
				Del enunciado tenemos:
				\begin{equation*}
					\begin{split}
						\vec{F}&=(2,-1,-1)\\
						\vec{r}&=(3,1,-5)\\
					\end{split}
				\end{equation*}
				Ahora calculando el trabajo:
				\begin{equation*}
					\begin{split}
						\vec{F}\cdot\vec{r}&=(6-1+5)=10
					\end{split}
				\end{equation*}
			\subsubsection{Demostraci\'on 3}
				Sea $\vec{A}=A_{1}\hat{i}+A_{2}\hat{j}+A_{3}\hat{k}$ cualquier vector.
				Demuestre que 
				$$
					\vec{A}=(\vec{A}\cdot\hat{i})\hat{i}+(\vec{A}\cdot\hat{j})\hat{j}
										+(\vec{A}\cdot\hat{k})\hat{k}
				$$
				\noindent\textsl{Soluci\'on:}\\
				Sabemos que
				\begin{equation}
					\begin{split}
						\vec{A}=A_{1}\hat{i}+A_{2}\hat{j}+A_{3}\hat{k} 
					\end{split}
					\label{VecA}
				\end{equation}
				Hacemos:
				\begin{equation*}
					\begin{split}
						\vec{A}\cdot\hat{i}&=A_{1}\\
						\vec{A}\cdot\hat{j}&=A_{2}\\
						\vec{A}\cdot\hat{i}&=A_{3}\\
					\end{split}
				\end{equation*}
				Sustituyendo $A_{1}$, $A_{2}$ y $A_{3}$ en \ref{VecA}:
				$$
					\vec{A}=(\vec{A}\cdot\hat{i})\hat{i}+(\vec{A}\cdot\hat{j})\hat{j}
										+(\vec{A}\cdot\hat{k})\hat{k}
				$$				
			\subsubsection{Diagonales de un rombo}
				Demuestre que las diagonales de un rombo son perpendiculares.\\
				\noindent\textsl{Soluci\'on:}\\
				Lo primero que hacemos es observar la figura, para demostrar que las diagonales del rombo
				son perpendiculares, tenemos que llegar a que:
				$$
					\vec{A}\cdot\vec{B}=0
				$$
				Observando la figura tenemos que:
				\begin{equation}
					\begin{split}
						\vec{L_{4}}+\vec{A}&=\vec{L_{1}}\\
						\therefore \vec{A}&=\vec{L_{1}}-\vec{L_{4}}\\
						\vec{B}&=\vec{L_{4}}+\vec{L_{3}}
					\end{split}
					\label{ApuntoB}
				\end{equation}
				Sustituyendo los resultados de la ecuaci\'on \ref{ApuntoB}:
				\begin{equation}
					\begin{split}
					\vec{A}\cdot\vec{B}&=(\vec{L_{1}}-\vec{L_{4}})\cdot
																(\vec{L_{4}}+\vec{L_{3}}) \\
															&=(\vec{L_{1}}\cdot\vec{L_{4}})+
																(\vec{L_{1}}\cdot\vec{L_{3}})-
																(\vec{L_{4}})^{2}-
																(\vec{L_{4}}\cdot\vec{L_{3}})
					\end{split}
					\label{L1L4}
				\end{equation}
				Por la figura sabemos que:
				\begin{equation*}
					\begin{split}
						\vec{L_{1}}&=\vec{L_{3}} \\
						\vec{L_{4}}&=\vec{L_{2}}
					\end{split}
				\end{equation*}
				Sustituimos $\vec{L_{1}}$ y $\vec{L_{4}}$ en las ecuaciones \ref{L1L4}:
				\begin{equation*}
					\begin{split} 
					\vec{A}\cdot\vec{B}&=(\vec{L_{1}}\cdot\vec{L_{2}})+
																(\vec{L_{1}}\cdot\vec{L_{1}})-
																(\vec{L_{4}})^{2}-
																(\vec{L_{2}}\cdot\vec{L_{1}})
					\end{split}
				\end{equation*}
				Por la propiedad conmutativa:
				$$
					\vec{L_{1}}\cdot\vec{L_{2}}=\vec{L_{2}}\cdot\vec{L_{1}}
				$$
				Entonces:
				\begin{equation*}
					\begin{split}
					\vec{A}\cdot\vec{B}&=(\vec{L_{1}}\cdot\vec{L_{2}})+
																(\vec{L_{1}}\cdot\vec{L_{1}})-
																(\vec{L_{4}^{2}})-
																(\vec{L_{1}}\cdot\vec{L_{2}}) \\
															&=(\vec{L_{1}}\cdot\vec{L_{1}})-
																(\vec{L_{4}^{2}})
					\end{split}
				\end{equation*}
				Recordando la propiedad:
				\begin{equation*}
					\begin{split}
						\vec{L_{1}}\cdot\vec{L_{1}}&=|\vec{L_{1}}|^{2}=(L_{1})^{2}\\
						\vec{L_{2}}\cdot\vec{L_{2}}&=|\vec{L_{2}}|^{2}=(L_{2})^{2}
					\end{split}
				\end{equation*}
				Entonces:
				$$	\vec{A}\cdot\vec{B}=(L_{1})^{2}-(L_{2})^{2} $$
				Y de la figura, sabemos que:
				$$ L_{1}=L_{2} $$
				Finalmente:
				$$ \vec{A}\cdot\vec{B}=0$$
				$\because$ las diagonales de un rombo son perpendiculares.

			\subsubsection{Vector unitario paralelo}
				\noindent\textsl{Soluci\'on:}\\
				Encuentre un vector unitario paralelo al plano $x$, $y$ y perpendicular al vector
				$\vec{A}=4\hat{i}-3\hat{j}+\hat{k}$ \\
				Para que $\vec{r}$ sea paralelo al plano $x$ y $y$ debe tener las componentes:
				$$
					\vec{r}=\alpha\hat{i}+\beta\hat{j}
				$$
				\noindentAhora para que sea perpendicular a $\vec{A}$ debe cumplir:
				$$
					\vec{A}\cdot\vec{r}=0
				$$
				\noindent Entonces sustituimos $\vec{A}$ y $\vec{r}$:
				\begin{equation*}
					\begin{split}
						\vec{A}\cdot\vec{r}&=(4,-3,1)\cdot(\alpha,\beta,0)\\
											&=4\alpha-3\beta=0\\
						\Rightarrow 4\alpha&=3\beta&
									&\alpha=\frac{3}{4}\beta
					\end{split}
				\end{equation*}
				Sustituyendo el valor que obtuvimos de $\alpha$ en $\vec{r}$ entonces:
				$$
					\vec{r}=\frac{3}{4}\beta\hat{i}+\beta\hat{j}
				$$
				Sin embargo, el problema nos pide un vector unitario, entonces:
				\begin{equation*}
					\begin{split}	
						\hat{e_{r}}&=\frac{\vec{r}}{|\vec{r}|}=\frac{\frac{3}{4}\beta\hat{i}+\beta\hat{j}}
																	{\sqrt{\frac{9}{16}\beta^{2}+\beta^{2}}}
									=\frac{\frac{3}{4}\beta\hat{i}+\beta\hat{j}}{\sqrt{\frac{25}{16}\beta^{2}}}
									=\pm\frac{\beta(\frac{3}{4}\hat{i}+\hat{j})}{\beta\frac{5}{4}}
									=\pm\frac{4(\frac{3}{4}\hat{i}+\hat{j})}{5}
									=\pm\frac{3\hat{i}+4\hat{j}}{5}
					\end{split}
				\end{equation*}

		% //////////////////////////////////////////////////
	    % /////// Cosenos directores de un vector /////////
	    % ////////////////////////////////////////////////
	    \subsection{C\'alculo de los cosenos directores de un vector}
	    Antes de empezar con el c\'alculo, es importante observar la figura para que sepas a que argumento
	    se va refiriendo.
	    Del dibujo sabemos que $\vec{a}$ tiene tres componentes: $(a_{1},a_{2},a_{3})$.\\
	    Entonces hacemos:
	    \begin{multicols}{3}
		    \columnseprulecolor{\color{black}}
		    \setlength{\columnseprule}{1pt}
	    	\begin{center}
		    \begin{equation*}
		    	\begin{split}
		   			 \vec{a}\cdot\hat{i}&=a\cdot1\cos\alpha\\
		   			 a_{1}&=a\cos\alpha
		    	\end{split}
		    \end{equation*}
		    Tenemos que:
		    $$\cos\alpha=\frac{a_{1}}{a}$$
		    \breakcolumn
		    \begin{equation*}
		    	\begin{split}
		   			 \vec{a}\cdot\hat{j}&=a\cdot1\cos\beta\\
		   			 a_{2}&=a\cos\beta
		    	\end{split}
		    \end{equation*} 
	   	    Tenemos que:
		    $$\cos\beta=\frac{a_{2}}{a}$$
		    \breakcolumn
		    \begin{equation*}
		    	\begin{split}
		   			 \vec{a}\cdot\hat{k}&=a\cdot1\cos\gamma\\
		   			 a_{3}&=a\cos\gamma
		    	\end{split}
		    \end{equation*} 
	   	    Tenemos que:
		    $$\cos\gamma=\frac{a_{3}}{a}$$
		    \end{center}
	    \end{multicols}
	  
	   	% /////////////////////////////////////
	    % /////// Producto vectorial /////////
	    % ///////////////////////////////////
	    \subsection{Producto vectorial}
	    \noindent\textsl{Definici\'on 1.}\\
	    Sean $\vec{A}$ y $\vec{B}$ dos vectores en $\mathbb{R}^{3}$ diferentes
	    del $\vec{0}$, entonces:
	    $$
	    	\vec{A}X\vec{B}=|\vec{A}||\vec{B}|\sin\theta\hat{u}
	    $$
	    donde:\\
	    $\theta$ se mide en sentido antihorario.\\
	    $\hat{u}$ es un vector unitario o perpendicular formado por $\vec{A}$ y $\vec{B}$.\\ \\
	    \noindent\textsl{Definici\'on 2.}\\
	    Si 
	    \begin{equation*}
	    	\begin{split}
	    		\vec{A}=A_{1}\hat{i}+A_{2}\hat{j}+A_{3}\hat{k}\\
	    		\vec{B}=B_{1}\hat{i}+B_{2}\hat{j}+B_{3}\hat{k}		
	    	\end{split}
	    \end{equation*}
	    entonces:
	    \begin{equation*}
	    	\begin{split}
	    		\vec{A}X\vec{B}=\begin{vmatrix}
	    							\hat{i} & \hat{j} & \hat{k} \\
	    							 A_{1}  &  A_{2}  & A_{3}   \\
	    							 B_{1}  &  B_{2}  & B_{3}   \\
 	    						\end{vmatrix}
	    	\end{split}
	    \end{equation*}
	    Esa es la definici\'on, pero veamos como se resuelve. Existen varios m\'etodos para obtener 
	    el determinante, el que veremos aqu\'i se llama \textsl{"Por menores"}. Entonces...
	    \begin{equation*}
	    	\begin{split}
	    		\vec{A}X\vec{B}&=\begin{vmatrix}
	    							\hat{i} & \hat{j} & \hat{k} \\
	    							 A_{1}  &  A_{2}  & A_{3}   \\
	    							 B_{1}  &  B_{2}  & B_{3}   \\
 	    						\end{vmatrix}
 	    						=(+)\hat{i} \begin{vmatrix}
 	    										A_{2} & A_{3} \\
 	    										B_{2} & B_{3} \\
 	    									\end{vmatrix}
 	    						+(-)\hat{j}\begin{vmatrix}
 	    										A_{1} & A_{3} \\
 	    										B_{1} & B_{3} \\
 	    									\end{vmatrix}
 	    						   +\hat{k}\begin{vmatrix}
 	    										A_{1} & A_{2} \\
 	    										B_{1} & B_{2} \\
 	    									\end{vmatrix}
	    	\end{split}
	    \end{equation*}
	    Y lo anterior hace que el problema de obtener el determinante sea menor, porque recordando:
	    \begin{equation*}
	   		\begin{split}
	    		det \bar{C}=\begin{vmatrix}
 	    						a & b \\
 	    						c & d \\
 	   						\end{vmatrix}
 	   					   =ad - cb  
	    	\end{split}
	    \end{equation*}
	    Entonces:
	    \begin{equation*}
	    	\begin{split}
	    		\vec{A}X\vec{B}&=\hat{i}(A_{2}B_{3}-A_{3}B_{2})
 	    						+(-)\hat{j}(A_{1}B_{3}-A_{3}B_{1})
 	    						   +\hat{k}(A_{1}B_{2}-A_{2}B_{1})
	    	\end{split}
	    \end{equation*}
	   	Ahora que ya sabemos qu\'e es el prodcuto vectorial, prestemos atenci\'on a la figura:
	    Es importante notar ciertos detalles con respecto a el producto cruz.
	    $$
	    		\vec{A}X\vec{B}\neq\vec{B}X\vec{A}
	    $$
	    De igual forma tenemos lo siguiente:
	    \begin{equation*}
	    	\begin{split}
	    		\vec{A}X\vec{A}=A^{2}\sin\theta \hat{u}=\vec{0}
	    	\end{split}
	    \end{equation*}
		% ////////////////////////////////////////////////////////////
	    % /////// Problemas clasicos del producto vectorial /////////
	    % //////////////////////////////////////////////////////////
		\subsection{Problemas Cl\'asicos del producto vectorial}

			\subsubsection{Ejercicio b\'asico}
				\noindent Suponga que $\vec{A}= \hat{i}-2\hat{j}-3\hat{k}$ y $\vec{B}=2\hat{i}+3\hat{j}-2\hat{k}$, 
				calcule |$\vec{A}X\vec{B}$| y |$(\vec{A}+2\vec{B})X(2\vec{A}-\vec{B})$|.\\
				\noindent\textsl{Soluci\'on:}\\
					\begin{itemize}
						\item |$\vec{A}X\vec{B}$|\\
							Primero hacemos el producto cruz:
								\begin{equation*}
									\begin{split}
										\vec{A}X\vec{B}&=\begin{vmatrix}
					 									\hat{i} & \hat{j} & \hat{k} \\
					 									   1    &   -2    &    -3   \\
					 									   2    &    3    &     -2   \\    
					 									\end{vmatrix}
					 									=+\hat{i} \begin{vmatrix}
					 	    										-2 & -3 \\
					 	    										 3 & -2 \\
					 	    									\end{vmatrix}
					 	    							-\hat{j}\begin{vmatrix}
					 	    										1 & -3 \\
					 	    										2 &  -2 \\
					 	    									\end{vmatrix}
					 	    						   +\hat{k}\begin{vmatrix}
					 	    										1 & -2 \\
					 	    										2 &  3 \\
					 	    									\end{vmatrix}\\
					 	    					   &=13\hat{i}-4\hat{j}+7\hat{k}
									\end{split}
								\end{equation*}
								 Finalmente:
								\begin{equation*}
								 	\begin{split}
								 		|\vec{A}X\vec{B}|=\sqrt{(13)^{2}+(-4)^{2}+(7)^{2}}=\sqrt{169+16+49}=\sqrt{234}
								 	\end{split}
								 \end{equation*}
						 	
						\item |$(\vec{A}+2\vec{B})X(2\vec{A}-\vec{B})$|\\
							Para facilitar las cosas primero haremos:
								\begin{equation*}
								 	\begin{split}
								 		\vec{A}+2\vec{B}&=\hat{i}-2\hat{j}-3\hat{k}+2(2\hat{i}+3\hat{j}-2\hat{k})   \\
								 						&=\hat{i}-2\hat{j}-3\hat{k}+4\hat{i}+6\hat{j}-4\hat{k}		\\
								 						&=5\hat{i}+4\hat{j}-7\hat{k}								\\
								 		2\vec{A}-\vec{B}&=2(\hat{i}-2\hat{j}-3\hat{k})-(2\hat{i}+3\hat{j}-2\hat{k})  \\
								 						&=2\hat{i}-4\hat{j}-6\hat{k}-2\hat{i}-3\hat{j}+2\hat{k}		 \\
								 						&=-7\hat{j}-4\hat{k}		 \\
								 	\end{split}
								 \end{equation*}
								Ahora hacemos el producto vectorial:
								\begin{equation*}
									\begin{split}
										(\vec{A}+2\vec{B})X(2\vec{A}-\vec{B})&=\begin{vmatrix}
											 									\hat{i} & \hat{j} & \hat{k} \\
											 									   5    &   4    &    -7   \\
											 									   0    &  -7    &    -4   \\    
											 									\end{vmatrix}
											 									=\hat{i} \begin{vmatrix}
											 	    										 4 & -7 \\
											 	    										-7 & -4 \\
											 	    									\end{vmatrix}
											 	    							-\hat{j}\begin{vmatrix}
											 	    										5 & -7 \\
											 	    										0 & -4 \\
											 	    									\end{vmatrix}
											 	    						   +\hat{k}\begin{vmatrix}
											 	    										5 & 4 \\
											 	    										0 & -7 \\
											 	    									\end{vmatrix}\\
											 	    					   &=65\hat{i}-20\hat{j}+35\hat{k} \\
											 	    					   &=5(13\hat{i}-4\hat{j}+7\hat{k})
									\end{split}
								\end{equation*}
								Y finalmente:
								\begin{equation*}
									\begin{split}
										|(\vec{A}+2\vec{B})X(2\vec{A}-\vec{B})|&=5\sqrt{(13)^{2}+(4)^{2}+(-7)^{2}} \\
																				&=5\sqrt{169+16+49}				   \\
																				&=5\sqrt{234}
									\end{split}
								\end{equation*}
					\end{itemize}
					
			\subsubsection{Ley de Senos}
				 Demostrar la Ley de los senos:
				 $$
				 	\frac{a}{\sen\alpha}=\frac{b}{\sen\beta}=\frac{c}{\sen\gamma}
				 $$
				 \noindent\textsl{Soluci\'on:}\\
				 De la figura tenemos que:
				 \begin{equation*}
				 	\begin{split}
				 		\vec{a}+\vec{b}+\vec{c}&=\vec{0} 
				 		&\Rightarrow \vec{a}=-\vec{b}-\vec{c} 
				 	\end{split}
				 \end{equation*}
				 Hacemos el producto vectorial de $\vec{a}$:
				 \begin{equation*}
				 		\vec{a}X\vec{a}=\vec{a}X(-\vec{b}-\vec{c})
				 \end{equation*}
				 Como el producto vectorial es distributivo:
				 \begin{equation*}
				 		\vec{0}=\vec{a}X\vec{a}=-(\vec{a}X\vec{b})-(\vec{a}X\vec{c})
				 \end{equation*}
				 Usando la \textsl{Definici\'on 1} del producto vectorial:
				 \begin{equation*}
				 		\vec{0}=-(ab\sen\gamma\hat{u_{1}})-(ac\sen\beta\hat{u_{2}})
				 \end{equation*}
				 Pero $\hat{u_{1}}=-\hat{u_{2}} $ entonces sustituimos $\hat{u_{1}}$
				 \begin{equation*}
				 	\begin{split}
				 		\vec{0}&=-ab\sen\gamma(-\hat{u_{2}})-ac\sen\beta(\hat{u_{2}}) \\
				 		0\hat{u_{2}}&=(ab\sen\gamma-ac\sen\beta)\hat{u_{2}}
				 	\end{split}
				 \end{equation*}
			    Recuerda... \textbf{"Dos vectores son iguales, si y solo si componente a componente son iguales"}, 
	    	    por lo tanto tenemos lo siguiente:
				\begin{equation*}
				 	\begin{split}
				 		0&=(ab\sen\gamma-ac\sen\beta)\\
				 		ab\sen\gamma&=ac\sen\beta \\
				 		b\sen\gamma&=c\sen\beta   \\
				 		\frac{b}{\sen\beta}&=\frac{c}{\sen\gamma}
				 	\end{split}
				 \end{equation*}

			\subsubsection{Demostraci\'on 1}
				\noindent Demuestra que $\sen(\alpha-\beta)=\sen\alpha\cos\beta-\cos\alpha\sin\beta$\\
				\noindent\textsl{Soluci\'on}:\\
				Observando el dibujo sabemos que:
				\begin{equation*}
					\begin{split}
						\vec{A}=A\cos\beta\hat{i}+A\sin\beta\hat{j}    \\
						\vec{B}=B\cos\alpha\hat{i}+B\sin\alpha\hat{j}  \\
					\end{split}
				\end{equation*}
				De la definici\'on 1 del producto vectorial, tenemos:\\
				\begin{equation}
					\begin{split}
						\vec{A}X\vec{B}=AB\sen(\alpha-\beta)\hat{k}
					\end{split}
					\label{Def_1}
				\end{equation}

				De la definici\'on 2 del producto vectorial, tenemos:\\
				\begin{equation}
					\begin{split}
						\vec{A}X\vec{B}&=\begin{vmatrix}
							   				  \hat{i}   &    \hat{j}    &  \hat{k} \\
											A\cos\beta  &  A\sin\beta   &     0    \\
											B\cos\alpha &  B\sin\alpha  &     0    \\
										\end{vmatrix}
										=\hat{k}\begin{vmatrix}
													A\cos\beta  &  A\sin\beta  \\
													B\cos\alpha &  B\sin\alpha \\
												\end{vmatrix}				   \\
									   &=\hat{k}(AB\sen\alpha\cos\beta-AB\cos\alpha\sin\beta) \\
									   &=AB\hat{k}(\sen\alpha\cos\beta-\cos\alpha\sin\beta)
					\end{split}
					\label{Def_2}
				\end{equation}
				Igualando las ecuaciones \ref{Def_1} y \ref{Def_2}:
				\begin{equation*}
				 		AB\sen(\alpha-\beta)\hat{k}&=AB(\sen\alpha\cos\beta-\cos\alpha\sin\beta)\hat{k}  
				 \end{equation*}
			    Recuerda... \textbf{"Dos vectores son iguales, si y solo si componente a componente son iguales"}, 
	    	    por lo tanto tenemos lo siguiente:
				 \begin{equation*}
				 	\begin{split}
				 		AB\sen(\alpha-\beta)&=AB(\sen\alpha\cos\beta-\cos\alpha\sin\beta)                \\
				 		\sen(\alpha-\beta)&=(\sen\alpha\cos\beta-\cos\alpha\sin\beta)                    \\
				 	\end{split}
				 \end{equation*}

			\subsubsection{Demostraci\'on 2}
				\noindent Demuestra que $\sen(\alpha+\beta)=\sen\alpha\cos\beta+\cos\alpha\sin\beta$\\
				\noindent\textsl{Soluci\'on}:\\
				Observando el dibujo sabemos que:
				\begin{equation*}
					\begin{split}
						\vec{A}=A\cos\beta\hat{i}-A\sin\beta\hat{j}    \\
						\vec{B}=B\cos\alpha\hat{i}+B\sin\alpha\hat{j}  \\
					\end{split}
				\end{equation*}
				De la definici\'on 1 del producto vectorial, tenemos:\\
				\begin{equation}
					\begin{split}
						\vec{A}X\vec{B}=AB\sen(\alpha+\beta)\hat{k}
					\end{split}
					\label{Def_1}
				\end{equation}

				De la definici\'on 2 del producto vectorial, tenemos:\\
				\begin{equation}
					\begin{split}
						\vec{A}X\vec{B}&=\begin{vmatrix}
							   				  \hat{i}   &    \hat{j}     &  \hat{k} \\
											A\cos\beta  &  -A\sin\beta   &     0    \\
											B\cos\alpha &  B\sin\alpha   &     0    \\
										\end{vmatrix}
										=\hat{k}\begin{vmatrix}
													A\cos\beta  &  -A\sin\beta  \\
													B\cos\alpha &   B\sin\alpha \\
												\end{vmatrix}				   \\
									   &=\hat{k}(AB\sen\alpha\cos\beta+AB\cos\alpha\sin\beta) \\
									   &=AB\hat{k}(\sen\alpha\cos\beta+\cos\alpha\sin\beta)
					\end{split}
					\label{Def_2}
				\end{equation}
				Igualando las ecuaciones \ref{Def_1} y \ref{Def_2}:
				\begin{equation*}
				 	\begin{split}
				 		AB\sen(\alpha+\beta)\hat{k}&=AB(\sen\alpha\cos\beta+\cos\alpha\sin\beta)\hat{k}  \\
				 	\end{split}
				 \end{equation*}
				Recuerda... \textbf{"Dos vectores son iguales, si y solo si componente a componente son iguales"}, 
	    	    por lo tanto tenemos lo siguiente:
				\begin{equation*}
				 	\begin{split}
				 		AB\sen(\alpha+\beta)&=AB(\sen\alpha\cos\beta+\cos\alpha\sin\beta)                \\
				 		\sen(\alpha+\beta)&=(\sen\alpha\cos\beta+\cos\alpha\sin\beta)             \\
				 	\end{split}
				 \end{equation*}

			\subsubsection{\'Area del paralelogramo}
			 	Calcule el \'area del paralelogramo cuyas diagonales son:
			 	\begin{equation*}
				 	\begin{split}
				 		\vec{A}&=3\hat{i}+\hat{j}-2\hat{k}\\
				 		\vec{B}&=\hat{i}-3\hat{j}-4\hat{k}
				 	\end{split}
				 \end{equation*}
				 \noindent\textsl{Soluci\'on:}\\
				 De la figura tenemos que:
				 \begin{multicols}{2}
					\begin{equation*}
					 	\begin{split}
					 		\vec{L_{1}}=\frac{1}{2}\vec{A}+\frac{1}{2}\vec{B}
					 	\end{split}
					 	\label{L_1}
					 \end{equation*}
				\breakcolumn
					 \begin{equation*}
					 	\begin{split}
					 		&\frac{1}{2}\vec{B}+\vec{L_{2}}=\frac{1}{2}\vec{A}              \\
					 		&\Rightarrow \vec{L_{2}}=\frac{1}{2}\vec{A}-\frac{1}{2}\vec{B}  \\
					 	\end{split}
					 	\label{L_2}
					 \end{equation*}
				 \end{multicols}
				Sustituimos lo valores de $\vec{A}$ y $\vec{B}$ en $\vec{L_{1}}$:
				\begin{equation*}
				 	\begin{split}
				 		\vec{L_{1}}=\frac{(3\hat{i}+\hat{j}-2\hat{k})+(\hat{i}-3\hat{j}-4\hat{k})}{2}  
				 					=\frac{4\hat{i}-2\hat{j}-6\hat{k}}{2}								
				 					=2\hat{i}-\hat{j}-3\hat{k}										    
				 	\end{split}
				 \end{equation*}
				Sustituimos lo valores de $\vec{A}$ y $\vec{B}$ en $\vec{L_{2}}$:
				\begin{equation*}
				 	\begin{split}
				 		\vec{L_{2}}=\frac{(3\hat{i}+\hat{j}-2\hat{k})-(\hat{i}-3\hat{j}-4\hat{k})}{2}  
				 					=\frac{2\hat{i}+4\hat{j}+2\hat{k}}{2} 									
				 					=\hat{i}+2\hat{j}+\hat{k}
				 	\end{split}
				 \end{equation*}
				Para obtener el \'area del paralelogramo, hacemos el producto vectorial de $\vec{L_{1}}$
				y $\vec{L_{2}}$:
				\begin{equation*}
				 	\begin{split}
				 		\vec{L_{1}}X\vec{L_{2}}&=\begin{vmatrix}
				 									\hat{i} & \hat{j} & \hat{k} \\
				 									   2    &   -1    &    -3   \\
				 									   1    &    2    &     1   \\    
				 								\end{vmatrix}
				 								=\hat{i} \begin{vmatrix}
				 	    										-1 & -3 \\
				 	    										 2 & 1 \\
				 	    									\end{vmatrix}
				 	    							-\hat{j}\begin{vmatrix}
				 	    										2 & -3 \\
				 	    										1 &  1 \\
				 	    									\end{vmatrix}
				 	    						   +\hat{k}\begin{vmatrix}
				 	    										2 & -1 \\
				 	    										1 &  2 \\
				 	    									\end{vmatrix}\\
				 	    					   &=5\hat{i}-5\hat{j}+5\hat{k}
				 	\end{split}
				 \end{equation*}
				 Finalmente:
				\begin{equation*}
				 	\begin{split}
				 		|\vec{L_{1}}X\vec{L_{2}}|=\sqrt{25+25+25}=\sqrt{75}=5\sqrt{3}
				 	\end{split}
				 \end{equation*}
			 
			\subsubsection{Vector perpendicular}
				Suponga que $\vec{A}=2\hat{i}+\hat{j}-3\hat{k}$ y $\vec{B}=\hat{i}-2\hat{j}+\hat{k}$. Encuentre
				un vector de magnitud 5 que sea perpendicular tanto a $\vec{A}$ como a $\vec{B}$.
				\noindent\textsl{Soluci\'on:}\\
				Para que el vector sea perpendicular hacemos el producto vectorial de $\vec{A}X\vec{B}$:
				\begin{equation*}
				 	\begin{split}
				 		\vec{A}X\vec{B}&=\begin{vmatrix}
				 									\hat{i} & \hat{j} & \hat{k} \\
				 									   2    &    1    &    -3   \\
				 									   1    &   -2    &     1   \\    
				 								\end{vmatrix}
				 								=\hat{i} \begin{vmatrix}
				 	    										  1 & -3 \\
				 	    										 -2 & 1 \\
				 	    									\end{vmatrix}
				 	    							-\hat{j}\begin{vmatrix}
				 	    										2 & -3 \\
				 	    										1 &  1 \\
				 	    									\end{vmatrix}
				 	    						   +\hat{k}\begin{vmatrix}
				 	    										2 &  1 \\
				 	    										1 & -2 \\
				 	    									\end{vmatrix}\\
				 	    					   &=-5\hat{i}-5\hat{j}-5\hat{k}
				 	\end{split}
				\end{equation*}
				Etiquetamos al vector resultante como $\vec{R}$ y obtenemos el m\'odulo:\\
				\begin{equation*}
				 	\begin{split}
				 		|\vec{R}|=\sqrt{25+25+25}=\sqrt{75}=5\sqrt{3}
				 	\end{split}
				\end{equation*}
				Primero obtendremos un vector unitario, y ya sabemos la definici\'on, por lo tanto:
				\begin{equation*}
				 	\begin{split}
				 		\hat{e_{r}}=\frac{-5\hat{i}-5\hat{j}-5\hat{k}}{5\sqrt{3}}		
				 					=\frac{5(-\hat{i}-\hat{j}-\hat{k})}{5\sqrt{3}}		
				 					=\frac{-\hat{i}-\hat{j}-\hat{k}}{\sqrt{3}}			
				 	\end{split}
				\end{equation*}
				El problema nos pide que sea de magnitud 5, por lo tanto multiplicamos el vector unitario por 5 
				y de esa forma queda resuelto el problema, siendo $\vec{X}$ el vector perpendicular a $\vec{A}$ y 
				$\vec{B}$:
				\begin{equation}
				 	\begin{split}
				 		\vec{X}=5\hat{e_{r}}=5\frac{-\hat{i}-\hat{j}-\hat{k}}{\sqrt{3}}
				 	\end{split}
				\end{equation}
		
		% //////////////////////////////////////////
	    % /////// Triple producto escalar /////////
	    % ////////////////////////////////////////
		\subsection{Triple producto escalar}
			Antes de empezar con la definici\'on, debes saber que tambi\'en  es conocido como \textsl{Producto caja}...\\
			Sean $\vec{A}$, $\vec{B}$ y $\vec{C}$ vectores distintos del $\vec{0}$, entonces:
			\begin{equation*}
		    	\begin{split}
		    		\vec{A}\cdot\vec{B}X\vec{C}=\begin{vmatrix}
					    							 A_{1}  &  A_{2}  & A_{3}   \\
					    							 B_{1}  &  B_{2}  & B_{3}   \\
					    							 C_{1}  &  C_{2}  & C_{3}   \\
				 	    						\end{vmatrix}
		    	\end{split}
		    \end{equation*}
		    Cuando $\vec{A}\cdot\vec{B}X\vec{C}=0$ se dice que los vectores son \textsl{coplanares}.\\
		    Cuando $\vec{A}\cdot\vec{B}X\vec{C}\neq0$ se dice que los vectores son \textsl{linealmente independientes}.\\
		    Sabemos que 3 vectores cuyo producto caja es 0, definen un plano.

	   	% ///////////////////////////////////////
	    % /////// Ecuacion de la recta  /////////
	    % //////////////////////////////////////
	    \subsection{Ecuaci\'on de la recta}

		    Tomando la vectorial de la recta y suponiendo que los puntos 
		    est\'an en $\mathbb{R}^3$, calcular la ecuaci\'on de la recta.
		    
		    $$ \vec{a} (1-\alpha)+ \alpha \vec{b} = \vec{r} $$
		  
		    \begin{multicols}{3}[Con]
		    $  \vec{a} = (x_{1}, y_{1}, z_{1}) $
	        \columnbreak
	        
		    $  \vec{b} = (x_{2}, y_{2}, z_{2})$
		   \columnbreak
		   
		   $   \vec{r} = (x, y, z)$
		    \end{multicols}
		    Empecemos... Lo primero que haremos es sustituir cada vector en 
		    la ecuaci\'on de la recta. 
		    $$
		    	(x_{1}, y_{1}, z_{1})(1-\alpha) + \alpha (x_{2}, y_{2}, z_{2}) = (x, y, z)
		    $$
	        Multiplicando y distribuyendo tenemos que
	        $$
	        	(x_{1}, y_{1}, z_{1})-(\alpha x_{1}, \alpha y_{1}, \alpha z_{1}) +
	        	(\alpha x_{2}, \alpha y_{2}, \alpha z_{2})= (x, y, z)
	        $$
	        Recuerda... \textbf{"Dos vectores son iguales, si y solo si componente a componente son iguales"} 
	        lo anterior implica lo siguiente:
	        \begin{equation}
	        	x_{1}- \alpha x_{1} + \alpha x_{2} = x
	        	\label{Ecuacion1} 
	        \end{equation}
	        \begin{equation}
	        	y_{1}- \alpha y_{1} + \alpha y_{2} = y
	        	\label{Ecuacion2}
	        \end{equation}
	        \begin{equation}
	        	z_{1}- \alpha z_{1} + \alpha z_{2} = z
	        	\label{Ecuacion3} 
	        \end{equation}        
	        De la ecuaci\'on \ref{Ecuacion1} tenemos:
	         \begin{equation*}
	         	\begin{split}
	        		x_{1}- \alpha x_{1} + \alpha x_{2} &= x 				\\
		      			  x_{1}+\alpha (x_{2} - x_{1}) &= x 				\\
		      			&\Rightarrow \alpha= \frac{x-x_{1}}{x_{2} - x_{1}}  \\
	        	\end{split}
	        \end{equation*}
	        De la ecuaci\'on \ref{Ecuacion2} tenemos:
	         \begin{equation*}
	         	\begin{split}
	        		y_{1}- \alpha y_{1} + \alpha y_{2} &= y 				\\
		      			  y_{1}+\alpha (y_{2} - y_{1}) &= y 				\\
		      			&\Rightarrow \alpha= \frac{y-y_{1}}{y_{2} - y_{1}} 	\\
	        	\end{split}
	        \end{equation*}
	        De la ecuaci\'on \ref{Ecuacion3} tenemos:
	         \begin{equation*}
	         	\begin{split}
	        		z_{1}- \alpha z_{1} + \alpha z_{2} &= z 				\\
		      			  z_{1}+\alpha (z_{2} - z_{1}) &= z 				\\
		      			&\Rightarrow \alpha= \frac{z-z_{1}}{z_{2} - z_{1}}  \\
	        	\end{split}
	        \end{equation*}
	        Entonces concluimos:
	        \begin{equation*}	
	        	\alpha=\frac{x-x_{1}}{x_{2} - x_{1}}=\frac{y-y_{1}}{y_{2} - y_{1}}
	        		  =\frac{z-z_{1}}{z_{2} - z_{1}}
	        \end{equation*}
	        Cabe mencionar que es buena idea memorizarla, ser\'a de ayuda para integrar.
		% ///////////////////////////////////////
	    % /////// Ecuacion de un plano /////////
	    % /////////////////////////////////////
		\subsection{Ecuaci\'on de un plano}
			Dados 3 puntos, hay un \'unico plano que los contiene. Por tanto la ecuaci\'on es:
			$$
				Ax+By+Cx=D
			$$
			Ahora, la ecuaci\'on vectorial del plano es:
			$$
				\vec{n}\cdot\vec{r}=0
			$$
	
		 	\subsubsection{Ejercicio}
			 	Describir el lugar geom\'etrico dado por |$\vec{r}-\vec{a}|=3$, donde $\vec{r}$ es el vector de 
			 	posici\'on de un punto en el espacio de coordenadas $(x,y,z)$ y $\vec{a}$ un vector de posici\'on
			 	de un punto en el espacio de coordenadas $(x_{1},y_{1},z_{1})$.\\
			 	\noindent\textsl{Soluci\'on:}\\
				\begin{equation*}
				 	\begin{split}
				 		\vec{r}-\vec{a}=((x-x_{1}),(y-y_{1}),(z-z_{1}))
				 	\end{split}
				\end{equation*}

				\begin{equation*}
				 	\begin{split}
				 		|\vec{r}-\vec{a}|&=\sqrt{(x-x_{1})^{2},(y-y_{1})^{2},(z-z_{1})^{2}}=3  	\\
				 						 &=(x-x_{1})^{2},(y-y_{1})^{2},(z-z_{1})^{2}=3^{2}
				 	\end{split}
				\end{equation*}
				El resultado es la ecuaci\'on de una esfera con centro en $(x_{1},y_{1},z_{1})$.

% ///////////////////////////////////////
% /// CAPITULO Derivacion vectorial ////
% /////////////////////////////////////
\chapter{Derivaci\'on vectorial}
		
	% ///////////////////////////////////////
	% /////// Derivacion vectorial /////////
	% /////////////////////////////////////
	\section{Derivaci\'on vectorial}
		Antes de empezar, debemos conocer a las funciones param\'etricas. En general una funci\'on parametrizada 
		se escribe de la siguiente forma:\\
		$$
			\phi(x(t),y(t),z(t))
		$$
		siendo $t$ su param\'etro.\\
		Ahora...\\
		\noindent\textsl{Definici\'on.}\\
		\begin{equation*}
			\begin{split}
				\frac{d\vec{A}(t)}{dt}=\frac{dA_{1}(t)}{dt}\hat{i}+\frac{dA_{2}(t)}{dt}\hat{j}
									   +\frac{dA_{3}(t)}{dt}\hat{k}
			\end{split}
		\end{equation*}

		% ///////////////////////////////////////
		% /////// Reglas de derivacion /////////
		% /////////////////////////////////////
		\subsection{Reglas de derivaci\'on}
			Obviamente no son todas, pero aqu\'i ver\'as algunas que te servir\'an por lo
		   	menos para sobrevivir...\\
				$$\frac{d}{dt}(\alpha\vec{A}+\beta\vec{B})=\alpha\frac{d\vec{A}}{dt}
																		 +\beta\frac{d\vec{B}}{dt}$$
				$$\frac{d}{dt}(\vec{A}\cdot\vec{B})=\vec{B}\cdot\frac{d\vec{A}}{dt}
																  +\vec{A}\cdot\frac{d\vec{B}}{dt}$$
				$$\frac{d}{dt}(\vec{A}X\vec{B})=\frac{d\vec{A}}{dt}X\vec{B}
										 						  +\vec{A}X\frac{d\vec{B}}{dt}$$	
		% //////////////////////////
		% /////// Notacion ////////
		% ////////////////////////
		\subsection{Notaci\'on}
			Existen m\'ultiples formas de escribir una derivada, he aqu\'i algunas:
			\begin{equation*}
				\begin{split}
					\frac{dy}{dx}&=y'														\\
					\frac{d^{2}y}{dx^{2}}&=\frac{d}{dx}\left(\frac{dy}{dy}\right)=y''		\\
					\frac{dy}{dt}&=\dot y 													\\
					\frac{d^{2}y}{dt^{2}}&=\ddot{y} 
				\end{split}
			\end{equation*}
				
		% ///////////////////////////////////////////////////
		% /////// Ejercicios basicos de derivacion /////////
		% /////////////////////////////////////////////////
		\subsection{Ejercicios b\'asicos de derivaci\'on}

			\subsubsection{Ejercicio 1}
				Si $\vec{A}(t)=e_{t}\hat{i}+\sin^{2}t\hat{j}+\cos^{2}t\hat{k}$ encuentra $\frac{d\vec{A}(t)}{dt}$.\\
				\noindent\textsl{Soluci\'on:}\\
				\begin{equation*}
					\begin{split}
						\frac{d\vec{A}(t)}{dt}=e_{t}\hat{i}+2\sin t\cos t\hat{j}-2\cos t\sin t\hat{k}
					\end{split}
				\end{equation*}

			\subsubsection{\¡Derivando\!}
				Suponga que $\vec{A}=t^{2}\hat{i}-t\hat{j}+(2t+1)\hat{k}$ y $\vec{B}=(2t-3)\hat{i}+\hat{j}-t\hat{k}$, con
				$t=1$.
				Encuentre: 
				\begin{multicols}{4}
					\begin{itemize}
						\item $\frac{d(\vec{A}\cdot\vec{B})}{dt}$
				\breakcolumn
						\item $\frac{d(\vec{A}X\vec{B})}{dt}$
		 		\breakcolumn
		 				\item $\frac{d(\vec{A}+\vec{B})}{dt}$
		 		\breakcolumn
		 				\item $\frac{d}{dt}\left(\vec{A}X\frac{d\vec{B}}{dt}\right)$
					\end{itemize}
				\end{multicols}

				\noindent\textsl{Soluci\'on:}
				\begin{itemize}
					\item $\frac{d(\vec{A}\cdot\vec{B})}{dt}$						\\
						Primero obtenemos $\vec{A}\cdot\vec{B}$:
						\begin{equation*}
							\begin{split}
								\vec{A}\cdot\vec{B}&=(2t-3)(t^{2})+(-t)+(2t+1)(-t)	\\
												   &=2t^{3}-3t^{2}-t-2t^{2}-t 		\\
												   &=2t^{3}-5t^{2}-2t
							\end{split}
						\end{equation*}
						Derivando:
						\begin{equation*}
							\begin{split}
								\frac{d(\vec{A}\cdot\vec{B})}{dt}&=\frac{d}{dt}(2t^{3}-5t^{2}-2t)\\
																 &=6t^{2}-10t-2
							\end{split}
						\end{equation*}
						Con $t=1$:
						\begin{equation*}
							\begin{split}
								\frac{d(\vec{A}\cdot\vec{B})}{dt}&=6-10-2=-6
							\end{split}
						\end{equation*}
					\item $\frac{d(\vec{A}X\vec{B})}{dt}$							\\
						Primero obtenemos $\vec{A}X\vec{B}$
						\begin{equation*}
							\begin{split}
								\vec{A}X\vec{B}&=\begin{vmatrix}
				 									\hat{i} & \hat{j} & \hat{k} 	\\
				 									   t^{2}  &   -t    &    2t+1   \\
				 									   2t-3   &    1    &     -t   	\\    
				 								\end{vmatrix}
				 								=\hat{i} \begin{vmatrix}
				 	    										-t & 2t+1 			\\
				 	    										 1 &  -t  			\\
				 	    									\end{vmatrix}
				 	    							-\hat{j}\begin{vmatrix}
				 	    										t^{2} & 2t+1 		\\
				 	    										2t-3  &  -t 		\\
				 	    									\end{vmatrix}
				 	    						   +\hat{k}\begin{vmatrix}
				 	    										t^{2} & -t 			\\
				 	    										2t-3  &  1 			\\
				 	    									\end{vmatrix}			\\
				 	    					   &=\left(t^{2}-(2t+1)\right)\hat{i}-\left(-t^{3}-(2t-3)(2t+1)\right)\hat{j}
				 	    					   	 +\left(t^{2}+t(2t-3)\right)\hat{k}						\\
				 	    					   &=(t^{2}-2t-1)\hat{i}-(-t^{3}-4t^{2}-2t+6t+3)\hat{j}
				 	    					   	 +(t^{2}+2t^{2}-3t)\hat{k}								\\
				 	    					   &=(t^{2}-2t-1)\hat{i}-(-t^{3}-4t^{2}+4t+3)\hat{j}
				 	    					   	 +(3t^{2}-3t)\hat{k}
							\end{split}
						\end{equation*}
						Derivando:
						\begin{equation*}
							\begin{split}
								\frac{d(\vec{A}X\vec{B})}{dt}&=\frac{d}{dt}\left((t^{2}-2t-1)\hat{i}-(-t^{3}-4t^{2}+4t+3)\hat{j}
				 	    					   	 			  +(3t^{2}-3t)\hat{k}\right) 		\\
				 	    					   	 			 &=(2t-2)\hat{i}-(-3t^{2}-8t+4)\hat{j}+(6t-3)\hat{k}
							\end{split}
						\end{equation*}
						Finalmente con $t=1$:
						\begin{equation*}
							\begin{split}
								\frac{d(\vec{A}X\vec{B})}{dt}&=(2-2)\hat{i}-(-3-8+4)\hat{j}+(6-3)\hat{k}\\
															 &=7\hat{j}+3\hat{k}
							\end{split}
						\end{equation*}
					\item $\frac{d(\vec{A}+\vec{B})}{dt}$\\
						Primero obtenemos $\vec{A}+\vec{B}$
						\begin{equation*}
							\begin{split}
								\vec{A}+\vec{B}&=t^{2}\hat{i}-t\hat{j}+(2t+1)\hat{k}+(2t-3)\hat{i}+\hat{j}-t\hat{k} \\
												&=(t^{2}+2t-3)\hat{i}+(1-t)\hat{j}+(t+1)\hat{k}						\\
							\end{split}
						\end{equation*}
						Derivando:
						\begin{equation*}
							\begin{split}
								\frac{d(\vec{A}+\vec{B})}{dt}&=\frac{d}{dt}\left((t^{2}+2t-3)\hat{i}+
																(1-t)\hat{j}+(t+1)\hat{k}\right)					\\
															 &=(2t+2)\hat{i}-\hat{j}+\hat{k}
							\end{split}
						\end{equation*}
						Finalmente con $t=1$
						\begin{equation*}
							\begin{split}
								\frac{d(\vec{A}+\vec{B})}{dt}&=(2+2)\hat{i}-\hat{j}+\hat{k}							\\
															 &=4\hat{i}-\hat{j}+\hat{k}
							\end{split}
						\end{equation*}

					\item $\frac{d}{dt}\left(\vec{A}X\frac{d\vec{B}}{dt}\right)$\\
					Primero hacemos $\frac{d\vec{B}}{dt}$
					\begin{equation*}
						\begin{split}
							\frac{d}{dt}(\vec{B})&=\frac{d}{dt}\left((2t-3)\hat{i}+\hat{j}-t\hat{k}\right)			\\
												 &=2\hat{i}-\hat{k}
						\end{split}
					\end{equation*}
					Hacemos el producto vectorial:
					\begin{equation*}
						\begin{split}
							\vec{A}X\frac{d\vec{B}}{dt}&=\begin{vmatrix}
						 									\hat{i} & \hat{j} & \hat{k} 	\\
						 									   t^{2}  &   -t    &    2t+1   \\
						 									     2    &    0    &     -1   	\\    
						 								\end{vmatrix}
						 								=\hat{i}\begin{vmatrix}
						 	    										-t & 2t+1 			\\
						 	    										 0 &  -1  			\\
						 	    								\end{vmatrix}			
						 	    						-\hat{j}\begin{vmatrix}
						 	    									t^{2} & 2t+1 		\\
						 	    									   2  &  -1 		\\
						 	    								\end{vmatrix}
						 	    						+\hat{k}\begin{vmatrix}
						 	    									t^{2} & -t 			\\
						 	    									  2   &  0 			\\
						 	    								\end{vmatrix}			\\
						 	    					   &=t\hat{i}-(-t^{2}-4t-2)\hat{j}+2t\hat{k}	\\
						\end{split}
					\end{equation*}
					Derivando:
					\begin{equation*}
						\begin{split}
							\frac{d}{dt}\left(\vec{A}X\frac{d\vec{B}}{dt}\right)&=\frac{d}{dt}\left(t\hat{i}-(-t^{2}-4t-2)
																				 \hat{j}+2t\hat{k}\right) 					\\
																				&=\hat{i}-(-2t-4)\hat{j}+2\hat{k}
						\end{split}
					\end{equation*}
					Finalmente con $t=1$
					\begin{equation*}
						\begin{split}
							\frac{d}{dt}\left(\vec{A}X\frac{d\vec{B}}{dt}\right)&=\hat{i}-(-2-4)\hat{j}+2\hat{k}		\\
																				&=\hat{i}+6\hat{j}+2\hat{k}
						\end{split}
					\end{equation*}

				\end{itemize}

			\subsubsection{Demostraci\'on 1}
				Demuestra que:
				$$
					\vec{A}\cdot\frac{d\vec{A}}{dt}=A\frac{dA}{dt}
				$$
				\textsl{Soluci\'on:}\\
				Sea 
				$$
					\vec{A}=A_{1}(t)\hat{i}+A_{2}(t)\hat{j}+A_{3}(t)\hat{k}
				$$
				entonces:
				$$
					|\vec{A}|=\sqrt{(A_{1})^{2}+(A_{2})^{2}+(A_{3})^{2}}
				$$
				Por un lado desarrollaremos la parte izquierda  $\vec{A}\cdot\frac{d\vec{A}}{dt}$:\\
				\begin{equation*}
					\begin{split}
						\vec{A}\cdot\frac{d\vec{A}}{dt}&=\vec{A}\cdot\left(\frac{dA_{1}}{dt}\hat{i}
														+\frac{dA_{2}}{dt}\hat{j}
														+\frac{dA_{3}}{dt}\hat{k}\right)			\\
													   &=\left(A_{1}\hat{i}+A_{2}\hat{j}+A_{3}\hat{k}\right)\cdot
													   	\left(\frac{dA_{1}}{dt}\hat{i}+\frac{dA_{2}}{dt}\hat{j}
															+\frac{dA_{3}}{dt}\hat{k}\right)
					\end{split}
				\end{equation*}
				Entonces:
				\begin{equation}
					\begin{split}
					\vec{A}\cdot\frac{d\vec{A}}{dt}=(A_{1})\frac{dA_{1}}{dt}+(A_{2})\frac{dA_{2}}{dt}
													+(A_{3})\frac{dA_{3}}{dt}
					\end{split}
					\label{parte1}
				\end{equation}
				Ahora desarrollamos el lado derecho $A\frac{dA}{dt}$:
				\begin{equation*}
					\begin{split}
						A\frac{dA}{dt}&=A\left(\frac{1}{2}(A_{1}^{2}+A_{2}^{2}+A_{3}^{2})^{-\frac{1}{2}}\right)
										\left(2A_{1}\frac{dA_{1}}{dt}+2A_{2}\frac{dA_{2}}{dt}+2A_{3}\frac{dA_{3}}{dt}\right)
					\end{split}
				\end{equation*}
				Entonces:
				\begin{equation}
					\begin{split}
					A\frac{dA}{dt}=(A_{1})\frac{dA_{1}}{dt}+(A_{2})\frac{dA_{2}}{dt}
													+(A_{3})\frac{dA_{3}}{dt}
					\end{split}
					\label{parte2}
				\end{equation}
				Como \ref{parte1}=\ref{parte2}, concluimos que:
				$$
					\vec{A}\cdot\frac{d\vec{A}}{dt}=A\frac{dA}{dt}
				$$
	
		% /////////////////////////////////////////////////////
		% /////// Ejercicios Ecuaciones diferenciales ////////
		% ///////////////////////////////////////////////////
		\subsection{Resolviendo ecuaciones diferenciales}
			
			\subsubsection{Ejercicio 1}
				\noindent Verifica que $\vec{r}=\vec{C_{1}}\sin2t$ es soluci\'on de $\frac{d^{2}\vec{r}}{dt^{2}}$.	\\
				\noindent\textsl{Soluci\'on:}\\
				Derivamos:
				\begin{equation*}
					\begin{split}
						\frac{d\vec{r}}{dt}&=2\vec{C_{1}}\cos2t	\\
					\end{split}
				\end{equation*}
				\begin{equation}
					\begin{split}
						\frac{d^{2}\vec{r}}{dt^{2}}&=\frac{d}{dt}\left(\frac{d\vec{r}}{dt}\right)	
												   =\frac{d}{dt}\left(2\vec{C_{1}}\cos2t\right)	
												   =-4\vec{C_{1}}\sin2t
					\end{split}
					\label{r2}
				\end{equation}
				Sustituyendo \ref{r2} en la ecuaci\'on diferencial:
				\begin{equation*}
					\begin{split}
						-4\vec{C_{1}}\sin2t+4\vec{C_{1}}\sin2t&=\vec{0} \\
					\end{split}
				\end{equation*}
				$\therefore$  $\vec{r}$ es soluci\'on de la ecuaci\'on diferencial.

			\subsubsection{Ejercicio 2}
				\noindent Sea $$\frac{d^{2}\vec{A}}{dt^{2}}=6t\hat{i} -24t^{2}\hat{j}+4\sint\hat{k}$$
				Encuentre $\vec{A}$ dado que:
				\begin{equation*}
					\begin{split}
						\vec{A}(t=0)=2\hat{i}+\hat{j}
					\end{split}
				\end{equation*}
				Y que con $t=0$:
				\begin{equation*}
					\begin{split}
						\frac{d\vec{A}}{dt}=-\hat{i}-3\hat{k}
					\end{split}
				\end{equation*}
				\noindent\textsl{Soluci\'on:}\\
				Sabemos que:
				\begin{equation*}
					\begin{split}
						\frac{d^{2}\vec{A}}{dt^{2}}=\frac{d}{dt}\left(\frac{d\vec{A}}{dt}\right)=*
					\end{split}
				\end{equation*}
				Observa lo siguiente:
				\begin{equation*}
					d\left(\frac{d\vec{A}}{dt}\right)=*dt
				\end{equation*}
				Entonces podemos decir que:
				\begin{equation}
					\begin{split}
						\frac{d\vec{A}(t)}{dt}=3t^{2}\hat{i}-8t^{3}\hat{j}-4\cost\hat{k}+\vec{C}
					\end{split}
					\label{E1}
				\end{equation}
				Evaluamos cuando $t=0$ en la ecuaci\'on \ref{E1} y con la condici\'on inicial del problema:
				\begin{equation*}
					\begin{split}
						\frac{d\vec{A}(t)}{dt}&=3(0)\hat{i}-8(0)\hat{j}-4\cost\hat{k}+\vec{C}				
											  =-\hat{i}-3\hat{k}											\\
											  &=-4\hat{k}+\vec{C}=-\hat{i}-3\hat{k}							\\
					\end{split}
				\end{equation*}
				Por lo tanto decimos:
				\begin{equation*}
					\begin{split}
						\Rightarrow \vec{C}&=-\hat{i}-3\hat{k}+4\cos t\hat{k}								\\
											&=-\hat{i}+(4-3)\hat{k}											\\
											&=-\hat{i}+\hat{k}	
					\end{split}
				\end{equation*}
				Sustituimos el valor del $\vec{C}$ en \ref{E1}:
				\begin{equation*}
					\begin{split}
						\frac{d\vec{A}(t)}{dt}&=3t^{2}\hat{i}-8t^{3}\hat{j}-4\cos t\hat{k}-\hat{i}+\hat{k}	\\
											  &=(3t^{2}-1)\hat{i}-8t^{3}\hat{j}+(1-4\cos t)\hat{k}
					\end{split}
				\end{equation*}
				Para obtener el valor de $\vec{A}$:
				\begin{equation*}
					\begin{split}
						\int\,\mathrm{d}\vec{A}&=\int\,\left((3t^{2}-1)\hat{i}-8t^{3}\hat{j}+(1-4\cos t)\hat{k}\right)\mathrm{d}t \\
									    \vec{A}&=\int\,(3t^{2}-1)\hat{i}\mathrm{d}t-\int\,8t^{3}\hat{j}\mathrm{d}t
											    +\int\,(1-4\cos t)\hat{k}\mathrm{d}t 											  \\
					\end{split}
				\end{equation*}
				Resolviendo las integrales:
				\begin{equation}
					\begin{split}
						\vec{A}&=(t^{3}-t)\hat{i}-2t^{4}\hat{j}+(t-4\sin t)\hat{k}+\vec{C_{1}}
					\end{split}
					\label{E2}
				\end{equation}
				Aplicando la condici\'on inicial $\vec{A}(t=0)=2\hat{i}+\hat{j}$:
				\begin{equation*}
					\begin{split}
						\vec{A}&=(0)\hat{i}-2(0)\hat{j}+(0)\hat{k}+\vec{C_{1}}=2\hat{i}+\hat{j}		\\
							   &\Rightarrow\vec{C_{1}}=2\hat{i}+\hat{j}
					\end{split}
				\end{equation*}
				Sustituimos el valor del $\vec{C_{1}}$ en \ref{E2}:
				\begin{equation*}
					\begin{split}
						\vec{A}&=(t^{3}-t)\hat{i}-2t^{4}\hat{j}+(t-4\sin t)\hat{k}+2\hat{i}+\hat{j}	\\
						\vec{A}&=(t^{3}-t+2)\hat{i}+(1-2t^{4})\hat{j}+(t-4\sin t)\hat{k}
					\end{split}
				\end{equation*}

			\subsubsection{Ecuaci\'on-Polinomio caracter\'istico}
				Resolver la ecuaci\'on diferencial donde $\alpha$ y $w$ son constantes.
				$$
					\frac{d^{2}\vec{r}}{dt^{2}}+2\alpha \frac{d\vec{r}}{dt}+ w^{2}\vec{r}=\vec{0}
				$$
				\noindent\textsl{Soluci\'on:}\\
				Proponemos que la soluci\'on sea $\vec{r}=\vec{C}e^{mt}$ donde $\vec{C}$ es un vector constante. As\'i que:
				\begin{equation*}
					\begin{split}
						\frac{d\vec{r}}{dt}&=\vec{C}me^{mt} \\
						\frac{d^{2}\vec{r}}{dt^{2}}&=\vec{C}m^{2}e^{mt}
					\end{split}
				\end{equation*}
				Sustituimos en la ecuaci\'on diferencial:
				\begin{equation*}
					\begin{split}
						\vec{C}m^{2}e^{mt}+2\alpha\vec{C}me^{mt}+w^{2}\vec{C}e^{mt}=\vec{0}
					\end{split}
				\end{equation*}
				Factorizamos:
				\begin{equation*}
					\begin{split}
						\vec{C}e^{mt}\left(m^{2}+2\alpha m+w^{2}\right)=\vec{0}
					\end{split}
				\end{equation*}
				Observamos que no podemos evaluar $\vec{C}e^{mt}$ porque queremos una soluci\'on \textbf{NO TRIVIAL} es decir,
				diferente de cero entonces $\vec{C}\neq0$ $\therefore$ evaluo al polinomio.			\\
				Bueno, a este polinomio se le conoce como: \textsl{\textbf{Polinomio caracteristico}}.	\\
				Buscamos el valor de $m$:
				\begin{equation*}
					\begin{split}
						m_{1,2}&=\frac{-b\pm\sqrt{b^{2}--4ac}}{2a} 		 		\\
							   &=\frac{-2\alpha\pm\sqrt{4\alpha^{2}-4w^{2}}}{2}	\\
							   &=-\alpha\pm\sqrt{\alpha^{2}-w^{2}} 
					\end{split}
				\end{equation*}
				Nuevamente nos detenemos y analizamos los resultados que surgen a partir del resultado obtenido de $m$:
				\begin{itemize}
					\item \textbf{CASO 1} $\alpha^{2}-w^{2}>0$ $\Rightarrow$ $m_{1}\neq m_{2}$  $\epsilon \mathbb{R}$ \\
						$$
							\vec{r}=\vec{C_{1}}e^{m_{1}t}+\vec{C_{2}}e^{m_{2}t}
						$$
					\item \textbf{CASO 2} $\alpha^{2}-w^{2}=0$ $\Rightarrow$ $m_{1}=m_{2}=m$  $\epsilon \mathbb{R}$ \\
						$$
							\vec{r}=\vec{C_{1}}e^{mt}+\vec{C_{2}}e^{mt}
						$$
					\item \textbf{CASO 3} $\alpha^{2}-w^{2}<0$ $\Rightarrow$ $m_{1}\neq m_{2}$  $\epsilon \mathbb{C}$ donde $\mathbb{C}$
								es el conjunto de n\'umeros complejos.\\
						En este caso las ra\'ices son:
						\begin{equation*}
							\begin{split}
								m_{1,2}&=-\alpha \pm \sqrt{\alpha^{2}-w^{2}}						\\
									   &=-\alpha \pm \sqrt{-(w^{2}-\alpha^{2})}						\\
									   &=-\alpha \pm \sqrt{(-1)}\sqrt{(w^{2}-\alpha^{2})}			\\
									   &=-\alpha \pm i\sqrt{w^{2}-\alpha^{2}} \text{ }\epsilon \mathbb{C}	\\
									   &=\alpha \pm i\beta
							\end{split}
						\end{equation*}
						Tendr\'iamos que:
						\begin{equation*}
							\begin{split}
								m_{1}&=\alpha +i\beta 	\\
								m_{2}&=\alpha -i\beta 	\\
							\end{split}						
						\end{equation*}
						La forma para escribir una ecuaci\'on diferencial cuyas ra\'ices son complejas es:
						\begin{equation*}
							\begin{split}
								\vec{r}&=\vec{C_{1}}e^{m_{1}t}+\vec{C_{2}}e^{m_{2}t}								\\
									   &=\vec{C_{1}}e^{(\alpha +i\beta)t}+\vec{C_{2}}e^{(\alpha -i\beta)t}			\\
									   &=\vec{C_{1}}e^{\alpha t}e^{i\beta t}+\vec{C_{2}}e^{\alpha t}e^{-i\beta t}	\\
									   &=\vec{C_{1}}e^{\alpha t}(\cos\beta t+i\sin \beta t)+
									     \vec{C_{2}}e^{\alpha t}(\cos\beta t-i\sin \beta t)							\\
									   &=e^{\alpha t}\left((\vec{C_{1}}+\vec{C_{2}})\cos\beta t
									   	+i(\vec{C_{1}}-\vec{C_{2}})\sin\beta t\right)
							\end{split}
						\end{equation*}
						Usamos $\vec{C_{1}}+\vec{C_{2}}=\vec{D_{1}}$ y $\vec{C_{1}}-\vec{C_{2}}=\vec{D_{2}}$:
						\begin{equation*}
							\begin{split}
								\vec{r}&=e^{\alpha t}\left(\vec{D_{1}}\cos\beta t
									   	+i\vec{D_{2}}\sin\beta t\right)
							\end{split}
						\end{equation*} 
				\end{itemize}

			\subsubsection{Ejercicio 3}
				Resolver:
				$$
					\frac{d^{2}\vec{r}}{dt^{2}}+16\vec{r}=\vec{0}
				$$
				\noindent\textsl{Soluci\'on:}\\
				Primero con el polinomio caracteristico buscamos las ra\'ices:
				\begin{equation*}
					\begin{split}
						m^{2}+16&=0	\\
							   m&=\pm 4i
					\end{split}
				\end{equation*}
				Como las ra\'ices son complejas, entonces recordamos que la soluci\'on debe estar dada por:
				$$	
					\vec{r}&=e^{\alpha t}\left(\vec{D_{1}}\cos\beta t +i\vec{D_{2}}\sin\beta t\right)
				$$
				Para este caso $\alpha=0$ y $\beta=4$ entonces la soluci\'on es:
				\begin{equation*}
					\begin{split}
						\vec{r}&=e^{(0)t}\left(\vec{D_{1}}\cos4 t +i\vec{D_{2}}\sin4 t\right)		\\
							   &=\vec{D_{1}}\cos4 t +i\vec{D_{2}}\sin4 t		
					\end{split}
				\end{equation*}

			\subsubsection{Ejercicio 4}
				Resolver:
				$$
					\frac{d^{2}\vec{r}}{dt^{2}}-4\frac{d\vec{r}}{dt}+ \vec{r}=\vec{0}
				$$
				\noindent\textsl{Soluci\'on:}\\
				Primero con el polinomio caracteristico buscamos las ra\'ices:
				\begin{equation*}
					\begin{split}
						m^{2}-&4m+1=0	\\
						m_{1,2}&=\frac{4 \pm \sqrt{16-4(1)(1)}}{2}					
							   =\frac{4 \pm \sqrt{16-4}}{2}							
							   =2\pm\frac{\sqrt{12}}{2}								
							   =2\pm\sqrt{3}
					\end{split}
				\end{equation*}
				Como las ra\'ices son diferentes, entonces recordamos que la soluci\'on debe estar dada por:
				$$	
				\vec{r}=\vec{C_{1}}e^{m_{1}t}+\vec{C_{2}}e^{m_{2}t}				
				$$
				Para este caso $m_{1}=2+\sqrt{3}$ y $m_{2}=2-\sqrt{3}$ entonces la soluci\'on es:
				\begin{equation*}
					\begin{split}
						\vec{r}&=\vec{C_{1}}e^{(2+\sqrt{3})t}+\vec{C_{2}}e^{2-\sqrt{3}t}			
					\end{split}
				\end{equation*}

			\subsubsection{Ejercicio 5}
				Resolver:
				$$
					\frac{d^{2}\vec{r}}{dt^{2}}+2\frac{d\vec{r}}{dt}+\vec{r}=\vec{0}
				$$
				\noindent\textsl{Soluci\'on:}\\
				Primero con el polinomio caracteristico buscamos las ra\'ices:
				\begin{equation*}
					\begin{split}
						m^{2}+2m+1&=0	\\
						(m+1)(m+1)&=0	\\
					\end{split}
				\end{equation*}
				Como las ra\'ices son iguales, entonces recordamos que la soluci\'on debe estar dada por:
				$$	
				\vec{r}=\vec{C_{1}}e^{mt}+\vec{C_{2}}e^{mt}			
				$$
				Para este caso $m_{1}=-1$ y $m_{2}=-1$ entonces la soluci\'on es:
				\begin{equation*}
					\begin{split}
						\vec{r}&=\vec{r}=\vec{C_{1}}e^{-t}+\vec{C_{2}}e^{-t}		
					\end{split}
				\end{equation*}

			\subsubsection{Ejercicio 6}
				Resolver:
				$$
					\frac{d^{2}\vec{r}}{dt^{2}}+4\frac{d\vec{r}}{dt}=\vec{0}
				$$
				\noindent\textsl{Soluci\'on:}\\
				Primero con el polinomio caracteristico buscamos las ra\'ices:
				\begin{equation*}
					\begin{split}
						m^{2}+4&=0		\\
						m^{2}&=-4		\\
						m^{2}&=\pm 2i	\\
					\end{split}
				\end{equation*}
				Como las ra\'ices son complejas, entonces recordamos que la soluci\'on debe estar dada por:
				$$	
					\vec{r}&=e^{\alpha t}\left(\vec{D_{1}}\cos\beta t +i\vec{D_{2}}\sin\beta t\right)
				$$
				Para este caso $\alpha=0$ y $\beta=2$ entonces la soluci\'on es:
				\begin{equation*}
					\begin{split}
						\vec{r}&=e^{(0)t}\left(\vec{D_{1}}\cos2 t +i\vec{D_{2}}\sin2 t\right)		\\
							   &=\vec{D_{1}}\cos2 t +i\vec{D_{2}}\sin2 t		
					\end{split}
				\end{equation*}


	% /////////////////////////////////////////
	% /////// Velocidad y aceleracion/////////
	% ///////////////////////////////////////
	\section{Velocidad y aceleraci\'on}
		Comenzaremos repasando ciertos conceptos que en un curso de f\'isica ya los aprendiste.\\
		\textbf{Desplazamiento}. Definimos el desplazamiento como el cambio en la posici\'on de un objeto. 
		El desplazamiento es una cantidad vectorial porque establece en que \textsl{direcci\'on} se mueve 
		el objeto. Usaremos $\vec{r}$ para identificarlo.\\
		\textbf{Velocidad}. En pocas palabras, es la relaci\'on  entre la distancia que recorre un objeto y 
		el tiempo que tarda en recorrerla. Suena l\'ogico entonces decir:
		$$
		   \frac{d\vec{r}}{dt}=\vec{v}=\mathnormal{velocidad}
		$$
		   \textbf{Rapidez}. Es la magnitud de la velocidad:
		$$
		   |\vec{v}|=\mathnormal{rapidez}
		$$
		Ahora, empecemos con un poco de explicaci\'on, supongamos que estas ...
			   %
			   % C  O  N  T  I  N  U  A  R
			   %
		% /////////////////////////////////////////
		% /////// Ejercicio de velocidad /////////
		% ///////////////////////////////////////
		\subsection{Ejercicio de velocidad}
			Suponga que una part\'icula se mueve a lo largo de la curva $x=2\sin3t$ , $y=2\cos3t$ y $z=8t$. 
			Encuentre la velocidad y aceleraci\'on de esta part\'icula para todo tiempo $>0$ y de igual forma
			su rapidez y celeridad para $t=\frac{\pi}{2}$\\
			\noindent\textsl{Soluci\'on:}\\
			El problema nos dice a lo largo de qu\'e curva se mueve la part\'icula, entonces usaremos un vector:
			$$
				\vec{A}=2\sin3t\hat{i}+2\cos3t\hat{j}+8t\hat{k}
			$$
			Calculamos la velocidad y aceleraci\'on para $t>0$:\\
			\begin{equation*}
				\begin{split}
					\vec{v}&=\frac{d\vec{A}}{dt}=6\cos3t\hat{i}-6\sin3t\hat{j}+8\hat{k} \\
					\vec{a}&=\frac{d^{2}\vec{A}}{dt^{2}}=-18\sin3t\hat{i}-18\cos3t\hat{j} 
				\end{split}
			\end{equation*}	
			Calculamos la velocidad y aceleraci\'on para $t=\frac{\pi}{2}$:
			\begin{equation*}
				\begin{split}
					\vec{v}&=\frac{d\vec{A}}{dt}=6\cos3\frac{\pi}{2}\hat{i}-6\sin3\frac{\pi}{2}\hat{j}+8\hat{k}\\
						   &=6\hat{j}+8\hat{k} \\
					\vec{a}&=\frac{d^{2}\vec{A}}{dt^{2}}=-18\sin3\frac{\pi}{2}\hat{i}-18\cos3\frac{\pi}{2}\hat{j}\\
						   &=18\hat{i}
				\end{split}
			\end{equation*}	
			Obteniendo la rapidez y celeridad:
			\begin{equation*}
				\begin{split}
					\mathnormal{rapidez}&=|\vec{v}|=\sqrt{36+64}=\sqrt{100}=10 \\
					\mathnormal{celeridad}&=|\vec{a}|=\sqrt{18^{2}}=18
				\end{split}
			\end{equation*}				

	% //////////////////////////////////////////////////////////////
	% /////// Derivadas parciales de funciones vectoriales ////////
	% ////////////////////////////////////////////////////////////
	\section{Derivadas parciales de funciones vectoriales}

		% /////////////////////////////
		% /////// Diferencial ////////
		% ///////////////////////////
		\subsection{Diferencial}
		\textsl{Definici\'on.}
		\begin{equation*}
			\begin{split}
				\mathrm{d}f(x,y,z)=\frac{\partial f(x,y,z)}{\partial x}\mathrm{d}x+\frac{\partial f(x,y,z)}{\partial y}\mathrm{d}y
								  +\frac{\partial f(x,y,z)}{\partial z}\mathrm{d}z
			\end{split}
		\end{equation*}

		% //////////////////////////
		% /////// Notacion ////////
		% ////////////////////////
		\subsection{Notaci\'on}
		Existen m\'ultiples formas de escribir una derivada parcial, he aqu\'i algunas:
			\begin{equation*}
				\begin{split}
					\frac{\partial f}{\partial x}&=f_{x}		\\
					\frac{\partial^{2} f}{\partial y \partial x}&=\frac{\partial}{\partial y }\left(\frac{\partial f}{\partial x}\right)
														=f_{xy}	\\
					\frac{\partial^{3} f}{\partial y \partial x^{2}}&=\frac{\partial}{\partial y}
																	 \left(\frac{\partial }{\partial x}
																	 \left(\frac{\partial f}{\partial x}\right)
																	 \right)=f_{xxy}
				\end{split}
			\end{equation*}

		% /////////////////////////////
		% /////// Propiedades ////////
		% ///////////////////////////
		\subsection{Propiedades}
		\begin{equation*}
			\begin{split}
				\frac{\partial}{\partial t}\left(\vec{A}\cdot\vec{B}\right)=\frac{\partial \vec{A}}{\partial t}\cdot\vec{B}
																		   +\vec{A}\cdot\frac{\partial \vec{B}}{\partial t}
			\end{split}
		\end{equation*}
		\begin{equation*}
			\begin{split}
				\frac{\partial}{\partial t}(\phi\vec{A})=\frac{\partial \phi}{\partial t}\vec{A}
															 +\phi\frac{\partial \vec{A}}{\partial t}
			\end{split}
		\end{equation*}
		\begin{equation*}
			\begin{split}
				\frac{\partial}{\partial t}\left(\vec{A}X\vec{B}\right)=\frac{\partial \vec{A}}{\partial t}X\vec{B}
																	   +\vec{A}X\frac{\partial \vec{B}}{\partial t}
			\end{split}
		\end{equation*}

		% ///////////////////////////
		% /////// Problemas ////////
		% /////////////////////////
		\subsection{Problemas}
			\subsubsection{Ejercicio 1}
				\noindent\textsl{Soluci\'on:}	\\
				\begin{equation*}
					\begin{split}
						\frac{\partial d}{\partial x}x^{2}\ln(3ye^{z})=\ln(3ye^{z})2x
					\end{split}
				\end{equation*}

			\subsubsection{Ejercicio 2}
				Si $\phi(x,y,z)=xy^{2}z$ y $\vec{A}=xz\hat{i}-xy^{2}\hat{j}+yz^{2}\hat{k}$. Hallar:
				$$
					\frac{\partial^{3} (\phi\vec{A})}{\partial x^{2}\partial z}
				$$
				\textsl{Soluci\'on:}		\\
				Primero realizamos:
				\begin{equation*}
					\begin{split}
						\phi\vec{A}=x^{2}y^{2}z^{2}\hat{i}-x^{2}y^{4}z\hat{j}+xy^{3}z^{3}\hat{k}
					\end{split}
				\end{equation*}
				Sabemos que:
				\begin{equation*}
					\begin{split}
						\frac{\partial^{3}(\phi\vec{A})}{\partial x^{2}\partial z}=
										\frac{\partial}{\partial x}\left
											(\frac{\partial}{\partial x}\left(\frac{\partial (\phi\vec{A})}{\partial z}\right)
										\right)
					\end{split}
				\end{equation*}
				Primero hacemos:
				\begin{equation*}
					\begin{split}
						\frac{\partial (\phi\vec{A})}{\partial z}&=\frac{\partial}{\partial z}(x^{2}y^{2}z^{2}\hat{i}-
																	x^{2}y^{4}z\hat{j}+xy^{3}z^{3}\hat{k}) \\
															     &=2x^{2}y^{2}z\hat{i}-x^{2}y^{4}\hat{j}+3xy^{3}z^{2}\hat{k}							     \\
					\end{split}
				\end{equation*}
				Entonces:
				\begin{equation*}
					\begin{split}
						\frac{\partial}{\partial x}\left(\frac{\partial (\phi\vec{A})}{\partial z}\right)
								&=\frac{\partial}{\partial x}\left(2x^{2}y^{2}z\hat{i}-x^{2}y^{4}\hat{j}+3xy^{3}z^{2}\hat{k}\right)	\\
								&=4xy^{2}z\hat{i}-2xy^{4}\hat{j}+3y^{3}z^{2}\hat{k}
					\end{split}
				\end{equation*}
				Finalmente:
				\begin{equation*}
					\begin{split}
							\frac{\partial^{3}(\phi\vec{A})}{\partial x^{2}\partial z}=4y^{2}z\hat{i}-2y^{4}\hat{j}
					\end{split}
				\end{equation*}
	
			\subsubsection{Ejercicio 3}
				Siendo $\vec{C_{1}}$ y $\vec{C_{2}}$ vectores constantes y $x$ un escalar constante demostrar
				que $\vec{H}=e^{-\lambda x}\left(\vec{C_{1}}\sin\lambda y + \vec{C_{2}}\cos\lambda y \right)$ es la soluci\'on de
				la ecuaci\'on diferencial en derivadas parciales:
				$$
					\frac{\partial^{2}H}{\partial x^{2}}+\frac{\partial^{2}H}{\partial y^{2}}=\vec{0}
				$$
				\noindent\textsl{Soluci\'on:}\\
				Para la soluci\'on de problemas parecidos a este, es necesario redefinir ciertas partes de la ecuaci\'on principal, nota que las
				redefiniciones dependen de una sola variable.
				Usaremos:
				\begin{equation*}
					\begin{split}
						\gamma&=e^{-\lambda x}		\\
						\vec{A}&=\vec{C_{1}}\sin\lambda y + \vec{C_{2}}\cos\lambda y
					\end{split}
				\end{equation*}
				Ya que redefinimos ahora si derivamos parcialmente:
				\begin{equation*}
					\begin{split}
						\vec{H_{x}}&=\vec{A}\left(-\lambda e^{-\lambda x}\right)		\\
						\vec{H_{xx}}&=\vec{A}\left(\lambda^{2} e^{-\lambda x}\right)
					\end{split}
				\end{equation*}
				Por  otro lado:
				\begin{equation*}
					\begin{split}
						\vec{H_{y}}&=\gamma\left(\lambda\vec{C_{1}}\cos\lambda y-\lambda\vec{C_{2}}\sin\lambda y\right)	\\
								   &=\gamma\lambda\left(\vec{C_{1}}\cos\lambda y-\vec{C_{2}}\sin\lambda y\right)		\\
						\vec{H_{yy}}&=\gamma\lambda\left(-\lambda\vec{C_{1}}\sin\lambda y-\lambda\vec{C_{2}}\cos\lambda y\right)	\\
									&=-\gamma\lambda^{2}\left(\vec{C_{1}}\sin\lambda y+\vec{C_{2}}\cos\lambda y\right)
					\end{split}
				\end{equation*}
				Sustituimos los resultados en la ecuaci\'on principal:
				\begin{equation*}
					\begin{split}
						&\vec{H_{xx}}+\vec{H_{yy}}=\vec{0}		\\
						\vec{A}\left(\lambda^{2} e^{-\lambda x}\right)&-\gamma\lambda^{2}\left(\vec{C_{1}}\sin\lambda y+\vec{C_{2}}\cos\lambda y\right)=\vec{0}
					\end{split}
				\end{equation*}
				Ahora sustituimos las redefiniciones que establecimos:
				\begin{equation*}
					\begin{split}
						\left(\vec{C_{1}}\sin\lambda y + \vec{C_{2}}\cos\lambda\right)\left(\lambda^{2} e^{-\lambda x}\right)
							-(e^{-\lambda x})\lambda^{2}\left(\vec{C_{1}}\sin\lambda y+\vec{C_{2}}\cos\lambda y\right)=\vec{0}	\\
						\lambda^{2} e^{-\lambda x}(\vec{0})=\vec{0}
					\end{split}
				\end{equation*}
				$\therefore$ $\vec{H}$ es soluci\'on de la ecuaci\'on diferencial.

			\subsection{Ejercicio 4}
				Siendo $\vec{A}=\cos xy\hat{i}+(3xy-2x^{3})\hat{j}-(3x+2y)\hat{k}$, hallar:\\
				\begin{multicols}{4}
					\begin{itemize}
						\item $\frac{\partial \vec{A}}{\partial x}$
				\breakcolumn
						\item $\frac{\partial \vec{A}}{\partial y}$
		 		\breakcolumn
		 				\item $\frac{\partial^{2} \vec{A}}{\partial y\partial x}$
		 		\breakcolumn
		 				\item $\frac{\partial^{2} \vec{A}}{\partial x\partial y}$
					\end{itemize}
				\end{multicols}
				\noindent \textsl{Soluci\'on:}
				\begin{equation*}
					\begin{split}
						\frac{\partial \vec{A}}{\partial x}&=\frac{\partial}{\partial x}
															\left(\cos xy\hat{i}+(3xy-2x^{3})\hat{j}-(3x+2y)\hat{k}\right)	\\
														   &=-y\sin xy\hat{i}+(3y-6x^{2})\hat{j}-3\hat{k}					\\
						\frac{\partial \vec{A}}{\partial y}&=\frac{\partial}{\partial y}
															\left(\cos xy\hat{i}+(3xy-2x^{3})\hat{j}-(3x+2y)\hat{k}\right)	\\
														   &=-x\sin xy\hat{i}+3x\hat{j}-2\hat{k}							\\
					\end{split}
				\end{equation*}
				Utilizando los resultados anteriores, calculamos los ejercicios sobrantes:				
				\begin{equation*}
					\begin{split}
						\frac{\partial^{2} \vec{A}}{\partial y\partial x}&=\frac{\partial}{\partial y}
									\frac{\partial \vec{A}}{\partial x}									
									=\frac{\partial}{\partial y}
									\left(-y\sin xy\hat{i}+(3y-6x^{2})\hat{j}-3\hat{k}\right)			
									=(-xy\cos xy-\sin xy)\hat{i}+3\hat{j}								\\
						\frac{\partial^{2} \vec{A}}{\partial x\partial y}&=\frac{\partial}{\partial x}
									\frac{\partial \vec{A}}{\partial y}									
									=\frac{\partial}{\partial x}
									\left(-x\sin xy\hat{i}+3x\hat{j}-2\hat{k}\right)					
									=(-xy\cos xy -\sin xy)\hat{i}+3\hat{j}	
					\end{split}
				\end{equation*}

				\begin{equation*}
					\begin{split}
					\end{split}
				\end{equation*}



	
% ///////////////////////////////////////////////////////
% /// CAPITULO Operador diferencial vectorial nabla ////
% /////////////////////////////////////////////////////
\chapter{Operador diferencial vectorial nabla}

	% //////////////////////////////////////////////////
	% //// El operador diferencial vectorial Nabla ////
	% ////////////////////////////////////////////////
	\section{El operador diferencial vectorial Nabla}

	   	% /////////////////////////////////////////////////////////////////////
	    % /////// Definicion en coordenadas cartesianas y propiedades ////////
	    % ///////////////////////////////////////////////////////////////////
	   \subsection{Definici\'on en coordenadas cartesianas y propiedades}

	% ///////////////////////
	% //// El gradiente ////
	% /////////////////////
	\section{El gradiente}

	   	% ///////////////////////////////////////////
	    % /////// Interpretacion geometrica ////////
	    % /////////////////////////////////////////
	   \subsection{Interpretaci\'on geom\'etrica}

	   	% ///////////////////////////////////////////
	    % /////// Aplicaciones y ejercicios ////////
	    % /////////////////////////////////////////
	   \subsection{Aplicaciones y ejercicios}

	% /////////////////////////
	% //// La divergencia ////
	% ///////////////////////
	\section{La divergencia}

	   	% /////////////////////////////////////////////////////////////////
	    % /////// Interpretacion geometrica y campos solenoidales ////////
	    % ///////////////////////////////////////////////////////////////
	   \subsection{Interpretaci\'on geom\'etrica y campos solenoidales}

	   	% ///////////////////////////////////////////
	    % /////// Aplicaciones y ejercicios ////////
	    % /////////////////////////////////////////
	   \subsection{Aplicaciones y ejercicios}

	% ////////////////////////
	% //// El rotacional ////
	% //////////////////////
	\section{El rotacional}

	   	% ///////////////////////////////////////////////////////////////////
	    % /////// Interpretacion geometrica y campos irrotacionales ////////
	    % /////////////////////////////////////////////////////////////////
	   \subsection{Interpretaci\'on geom\'etrica y campos irrotacionales}

	   	% ///////////////////////////////////////////
	    % /////// Aplicaciones y ejercicios ////////
	    % /////////////////////////////////////////
	   \subsection{Aplicaciones y ejercicios}

% /////////////////////////////////////////////
% /// CAPITULO Calculo integral vectorial ////
% ///////////////////////////////////////////
\chapter{C\'alculo integral vectorial}

	% ////////////////////////////////////////////////
	% //// Coordenadas curvilineas generalizadas ////
	% //////////////////////////////////////////////
	\section{Coordenadas curvil\'ineas generalizadas}

	   	% //////////////////////////////////////////////
	    % /////// Ecuaciones de transformacion ////////
	    % ////////////////////////////////////////////
	   \subsection{Ecuaciones de transformaci\'on}

	   	% ///////////////////////////////////////////////////////////
	    % /////// Curvas coordenadas y syperficies de nivel ////////
	    % /////////////////////////////////////////////////////////
	   \subsection{Curvas coordenadas y superficies de nivel}

		% ////////////////////////////////////////////////////////////////
	    % /////// Elementos de linea, de superficie y de volumen ////////
	    % //////////////////////////////////////////////////////////////
	   \subsection{Elementos de l\'inea, de superficie y de volumen}

	   	% /////////////////////////////////////////////////////////////////
	    % /////// Gradiente, divergencia, rotacional y laplaciano ////////
	    % ///////////////////////////////////////////////////////////////
	   \subsection{Gradiente, divergencia, rotacional y laplaciano}

	   	% ///////////////////////////////////////////////////////////////////
	    % /////// Aplicacion a: coordenadas cilindricas y esfericas ////////
	    % /////////////////////////////////////////////////////////////////
	   \subsection{Aplicaci\'on a coordenadas cil\'indricas y esf\'ericas}

	% ////////////////////////////
	% //// Integral de linea ////
	% //////////////////////////
	\section{Integral de l\'inea}

		% /////////////////////////////////////////
	    % /////// Definicion y ejercicios ////////
	    % ///////////////////////////////////////
	   \subsection{Definici\'on y ejercicios}

		% /////////////////////////////
	    % /////// Propiedades ////////
	    % ///////////////////////////
	   \subsection{Propiedades}

		% /////////////////////////////////////////////////
	    % /////// Teorema de campos conservativos ////////
	    % ///////////////////////////////////////////////
	   \subsection{Teorema de campos conservativos}

	% //////////////////////////////////////
	% //// Integrales dobles y triples ////
	% ////////////////////////////////////
	\section{Integrales dobles y triples}

		% /////////////////////////////////////
	    % /////// Integrales iteradas ////////
	    % ///////////////////////////////////
	   \subsection{Integrales iteradas}

	% ///////////////////////////////////
	% //// Integrales de superficie ////
	% /////////////////////////////////
	\section{Integrales de superficie}

		% //////////////////////////////////////////////////////////////
	    % /////// Integrales de superficie de un campo escalar ////////
	    % ////////////////////////////////////////////////////////////
	   \subsection{Integrales de superficie de un campo escalar}

		% ////////////////////////////////////////////////////////////////////////////////////////////
	    % /////// Integrales de superficie de un campo vectorial e interpretación geometrica ////////
	    % //////////////////////////////////////////////////////////////////////////////////////////
	   \subsection{Integrales de superficie de un campo vectorial e interpretaci\'on geom\'etrica}

	% ////////////////////////////////
	% //// Integrales de volumen ////
	% //////////////////////////////
	\section{Integrales de volumen}

		% ///////////////////////////////////////////////////
		% //// Integrales de volumen de un campo escalar////
		% /////////////////////////////////////////////////
		\subsection{Integrales de volumen de un campo escalar}

		% //////////////////////////////////////////////////////
		% //// Integrales de volumen de un campo vectorial ////
		% ////////////////////////////////////////////////////
		\subsection{Integrales de volumen de un campo vectorial}

	% //////////////////////////////
	% //// Teoremas integrales ////
	% ////////////////////////////
	\section{Teoremas integrales}

		% /////////////////////////////////////////////////////////////////////////
	    % /////// Teorema de Stokes, interpretacion fisica y aplicaciones ////////
	    % ///////////////////////////////////////////////////////////////////////
	   \subsection{Teorema de Stokes, interpretaci\'on f\'isica y aplicaciones}
		
		% /////////////////////////////////////////////////////////////
	    % /////// Teorema de Green en el plano y aplicaciones ////////
	    % ///////////////////////////////////////////////////////////
	   \subsection{Teorema de Green en el plano y aplicaciones}

		% ////////////////////////////////////////////////////////////////////////
	    % /////// Teorema de Gauss, interpretacion fisica y aplicaciones ////////
	    % /////////////////////////////////////////////////////////////////////
	   \subsection{Teorema de Gauss, interpretaci\'on f\'isica y aplicaciones}

\end{document}